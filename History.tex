\mysection{History of Ice Ages}
	\mysub{Ice ages}
		\ddd An ice age is a long interval of time (millions to tens of millions of years) when global temperatures are relatively cold and large areas of the Earth are covered by continental ice sheets and alpine glaciers. Within an ice age are multiple shorter-term periods of warmer temperatures when glaciers retreat (called interglacials or interglacial cycles) and colder temperatures when glaciers advance (called glacials or glacial cycles).
		\ddd At least five major ice ages have occurred throughout Earth’s history: the earliest was over 2 billion years ago, and the most recent one began approximately 3 million years ago and continues today (yes, we live in an ice age!).
		\ddd Currently, we are in a warm interglacial that began about 11,000 years ago. The last period of glaciation, which is often informally called the “Ice Age,” peaked about 20,000 years ago. At that time, the world was on average probably about $ 10^\circ F (5^\circ C) $ colder than today, and locally as much as $ 40^\circ F (22^\circ C) $ colder.
	\mysub{What causes ice ages?}
		\ddd Many factors contribute to climate variations, including changes in ocean and atmosphere circulation patterns, varying concentrations of atmospheric carbon dioxide, and even volcanic eruptions. The following discusses key factors in (1) initiating ice ages and (2) the timing of glacial-interglacial cycles.
		\ddd One significant trigger in initiating ice ages is the changing positions of Earth’s ever-moving continents, which affect ocean and atmospheric circulation patterns. When plate-tectonic movement causes continents to be arranged such that warm water flow from the equator to the poles is blocked or reduced, ice sheets may arise and set another ice age in motion.
		\ddd Today’s ice age most likely began when the land bridge between North and South America (Isthmus of Panama) formed and ended the exchange of tropical water between the Atlantic and Pacific Oceans, significantly altering ocean currents.
	\mysub{How does ice build up?}
		\ddd Throughout the Quaternary period, high latitude winters have been cold enough to allow snow to accumulate. It is when the summers are cold, (i.e., summers that occur when the sun is at its farthest point in Earth's orbit), that the snows of previous winters do not melt completely. When this process continues for centuries, ice sheets begin to form. Finally, the shape of Earth's orbit also changes. At one extreme, the orbit is more circular, so that each season receives about the same amount of insolation. At the other extreme, the orbital ellipse is stretched longer, exaggerating the differences between seasons. The eccentricity of Earth's orbit also proceeds through a long cycle, which takes 100,000 years. Major glacial events in the Quaternary have coincided when the phases of axial tilt, precession of equinoxes and eccentricity of orbit are all lined up to give the northern hemisphere the least amount of summer insolation.
	\mysub{Karoo Ice Age}
		Major Late Paleozoic ice age caused by ocean current disruption and named after a South African basin
	\mysub{Glacial History of Quaternary}
		The Quaternary System is that lasted from the present to approximately 2.588 million years ago with the Neogene system before the Quaternary. The Quaternary System contains two series: the Holocene and the Pleistocene with the Holocene being the present. In this period, ice sheets were able to form in Greenland and Antarctica and the continents were formed to their present shape. As glaciers formed and later retreated, thousands of lakes and rivers were created all over the world. As the glaciers retreated the sea level rose and the amount of biological diversity in the oceans increased
		\mysub{Wisconsin}
		This Wisconsin glaciation left widespread impacts on the North American landscape. The Great Lakes and the Finger Lakes were carved by ice deepening old valleys. Most of the lakes in Minnesota and Wisconsin were gouged out by glaciers and later filled with glacial meltwaters. The old Teays River drainage system was radically altered and largely reshaped into the Ohio River drainage system. Other rivers were dammed and diverted to new channels, such as Niagara Falls, which formed a dramatic waterfall and gorge, when the waterflow encountered a limestone escarpment. Another similar waterfall, at the present Clark Reservation State Park near Syracuse, New York, is now dry.
 \mysection{History of Events: }
    		\mysub{Glacial Buzzsaw Hypothesis}
    			This is the theory that the maximum height of mountain ranges is controlled by the erosion by glaciers. Evidence for: mountains are higher closer to the tropics where there is less glaciation, many mountain complexes are oriented towards the ELA line. Evidence against: modern studies of erosion rates do not correlate, cirques do not totally flatten the landscape. 
    		\mysub{Indicative of glacial advance:} 
    			moraine, horn, arête, cirque, erratic (think erosional features); indicative of retreat – moraine, tor, kame Sedimentary Deposits Rhythmites - are deposits of sedimentary rock that occur with periodicity and cyclicity. Can be indicative of events occurring on annual or 9longer scales (i.e. rise and fall of sea level, sediments pulsing out to ocean, etc). 
    		\mysub{Cooling of Antarctica -} 
    			The opening of the Drake Passage and formation of the antarctic circumpolar current. ~50 mya onset of glaciation around 3 mya - Closure of the Panama Isthmus, therefore increasing the intensity of North Atlantic Deep Water formation thereby cooling higher latitudes. Uplift of the Tibetan plateau (increased silicate weathering leading to CO2 drawdown and cooling) . Deepening of the Bering Strait allowing more cold Arctic water to flow through. 
    		\mysub{Younger Dryas Event}
    			This was an abrupt cooling that occurred in the middle of the current warming period ~24-26ka. Cooling is thought to be linked to changes in ocean circulation, with temperature drops of 2-6 degrees Celsius. It is often correlated to the onset of agriculture in human civilization.
    		\mysub{Snowball Earth: }
    			\textbf{2.4 to 2.1 billion years ago} \ddd
    			\vocab{The Huronian glaciation} is the oldest ice age we know about. The Earth was just over 2 billion years old, and home only to unicellular life-forms
    			The early stages of the Huronian, from 2.4 to 2.3 billion years ago, seem to have been particularly severe, with the entire planet frozen over in the first “snowball Earth”. This may have been triggered by a 250-million-year lull in volcanic activity, which would have meant less carbon dioxide being pumped into the atmosphere, and a reduced greenhouse effect.
    		\mysub{Evidence for Snowball Earth:} 
    			\ddd Glacial deposits found at tropical latitudes, very extensive and thick glacial deposits, negative carbon isotope excursions, banded iron formations. After the event we would expect to find cap carbonates, that would have formed as a result of rapidly changing ocean chemistry and the quick precipitation of calcium carbonate. Cyclothems = sea level periodically rises and falls.
    		\mysub{Evidence Against Snowball Earth: }
    			(1) Pulses of continental drift formed large, tropical plateaus, supporting glaciation; (2) Earth’s axial tilt was naturally very high during the Proterozoic, supporting mid-latitude and tropical glaciation (for more information, see the Zipper rift hypothesis, high-obliquity hypothesis, and polar wander)
    		\mysub{Zipper Rift Hypothesis}
    			The "Zipper rift" hypothesis proposes two pulses of continental "unzipping"—first, the breakup of the supercontinent Rodinia, forming the proto-Pacific Ocean; then the splitting of the continent Baltica from Laurentia, forming the proto-Atlantic—coincided with the glaciated periods.
    		\mysub{Deep Freeze: }
    			\textbf{850 to 630 million years ago}
    			\ddd During the 200 million years of the Cryogenian period, the Earth was plunged into some of the deepest cold it has ever experienced – and the emergence of complex life may have caused it.
    			\ddd One theory is that the glaciation was triggered by the evolution of large cells, and possibly also multicellular organisms, that sank to the seabed after dying. This would have sucked CO2 out of the atmosphere, weakening the greenhouse effect and thus lowering global temperatures
    			\ddd There seem to have been two distinct Cryogenian ice ages: the so-called Sturtian glaciation between 750 and 700 million years ago, followed by the Varanger (or Marinoan) glaciation, 660 to 635 million years ago. There’s some evidence that Earth became a snowball at times during the big freezes, but researchers are still trying to work out exactly what happened.
 			\mysub{Mass Extinction: }
 				\textbf{460 to 430 million years ago}
 				\ddd Straddling the late Ordovician period and the early Silurian period, the Andean-Saharan ice age was marked by a mass extinction, the second most severe in Earth’s history.
 				\ddd The die-off was surpassed only by the gargantuan Permian extinction 250 million years ago. But as the ecosystem recovered after the freeze, it expanded, with land plants becoming common over the course of the Silurian period. And those plants may have caused the next great ice age.
		\mysub{Plants Invade the Land: }
			\textbf{360 to 260 million years ago: }
			\ddd Like the Cryogenian glaciation, the \vocab{Karoo} ice age featured two peaks in ice cover that may well have been distinct ice ages. They took place in the Mississipian period, 359 to 318 million years ago, and again in the Pennsylvanian 318 to 299 million years ago.
			\ddd These ice ages may have been the result of the expansion of land plants that followed the Cryogenian. As plants spread over the planet, they absorbed CO2 from the atmosphere and released oxygen. As a result CO2 levels fell and the greenhouse effect weakened, triggering an ice age.
			\ddd There is some evidence that the ice came and went in regular cycles, driven by changes in Earth’s orbit. If true, this would mean that the Karoo ice age operated in much the same way as the current one
		\mysub{Antartica Freezes Over: }
			\textbf{14 million years ago}
			\ddd Antarctica wasn’t always frozen. It wasn’t until around 34 million years ago that the first small glaciers formed on the tops of Antarctica’s mountains. And it was 20 million years later, when world-wide temperatures dropped by $ 8^\circ C $, that the glaciers’ ice froze onto the rock, and the southern ice sheet was born.
			\ddd This temperature drop was triggered by the rise of the Himalayas. As they grew higher they were exposed to increased weathering, which sucked CO2 out of the atmosphere and reduced the greenhouse effect.
			\ddd The northern hemisphere remained relatively ice-free for longer, with Greenland and the Arctic becoming heavily glaciated only around 3.2 million years ago.
		\mysub{Our Ice Age: }
		\ddd First, the chilly “Older Dryas” of 14,700 to 13,400 years ago transformed most of Europe from forest to tundra, like modern-day Siberia. After a brief respite, the Younger Dryas, between 12,800 to 11,500 years ago, froze Europe solid within a matter of months – probably as a result of meltwater from retreating glaciers shutting down the Atlantic Ocean’s “conveyor-belt” current, although a cometary impact has also been blamed.
		\ddd Twelve thousand years ago, the great ice sheets retreated at the beginning of the latest interglacial – the Flandrian – allowing humans to return to northern latitudes. This period has been relatively warm, and the climate relatively stable, although it has been slightly colder than the last interglacial, the Eemian, and sea levels are currently at least 3 metres lower – differences that are being closely scrutinised by researchers keen to understand how our climate will develop.
 \mysection{Glacier Fluctuations}
     \ddd In ~1930 Milutin Milankovitch proposed that variations in three parameters of the earth's orbit caused glacial fluctuations: 
     \ddd 1.	Orbital eccentricity - the orbit of the earth around the sun is not a circle, but is elliptical and also varies. This eccentricity is a minor cause for seasons.
     \ddd 2.	Tilt variations in the axis of rotation (obliquity) - the tilt of the earth's rotational axis varies with time. A tilted axis is the primary cause of seasons. This varies between 22.1 and 24.5 in a 40,000 year cycle
     \ddd 3.	Precession - the earth's axis of rotation wobbles which results in minor fluctuations in the amount of solar radiation we receive.
     \ddd Milankovitch pacing seems to best explain glaciation events with periodicity of \blue{(Eccentricity) 100k, (Obliquity) 41,000, and (Axial Precession) 25,771.5 years.} This pattern seems to fit the info on climate change found in oxygen isotope cores. However, there are some problems with the Milankovitch theories.
     \ddd \vocab{\textbf{100,000 year Problem eccentricity}} variations have a significantly smaller impact on solar forcing than precession or obliquity and may be expected to produce the weakest effects. The greatest observed response is at the 100k year timescale, while the theoretical forcing is smaller at this scale, in regard to the ice ages. During the last 1 million years, the strongest climate signal is the 100k year cycle.
     \ddd \vocab{\textbf{400,000 year Problem}} (aka stage 11 problem) eccentricity variations have a strong 400k year cycle. That cycle is only clearly present in climate records older than the last million years.
     \ddd \vocab{Stage 5} problem refers to the timing of the penultimate interglacial that appears to have begun 10k years in advance of the solar forcing hypothesized to have caused it.