\mysection{Where are glaciers found?}
            \ddd Antarctica
            \ddd Greenland:	1,784,000
            \ddd Canada: 200,000
            \ddd Central Asia:	109,000
            \ddd Russia:	82,000
            \ddd United States:	75,000 (including Alaska)
            \ddd China and Tibet:	33,000
            \ddd South America: 25,000
            \ddd Iceland:	11,260
            \ddd Scandinavia:	2,909
            \ddd Alps: 2,900
            \ddd New Zealand:	1,159
            \ddd Mexico: 11
            \ddd Indonesia:	7.5
            \ddd Africa: 10
        \mysection{Famous Scientists}
        	\blue{Louis Agassiz} – He developed Continental Glaciation, and his hypothesis was that much of the continent of North America was covered by glacial ice that was 2 miles thick and which extended over much of the midwest. In 1840, Agassiz published a two-volume work entitled Études sur les glaciers. First to scientifically propose the existence of past ice ages. 
        	\blue{Milankovic} – he explained the Earth’s long term climate changes caused by changes in the position of the Earth in comparison to the Sun, now known as Milankovitch Cycles. This explained the ice ages occurring in the geological past of the Earth, as well as the climate changes on the Earth which can be expected in the future. 
        	\blue{Jens Esmark} – extension of past glaciations 
        	\blue{James David Forbes} – concluded that glaciers were viscous 
        	\blue{Louis Lilboutry} - Formation of penitentes, surveyed Andean glaciers 
        	\blue{Mark Meier} - Expert on sea level rise due to melting glaciers; Director of the Institute of Arctic and Alpine Research (INSTAAR) from 1985 to 1994. 
        	\blue{Louis Rendu} - Theorized on glacier motion 
        	\blue{Valter Schytt} - Studied Storglaciären in northern Sweden 
        	\blue{Wilhelm Sievers} – documented south American ice ages 
        	\blue{John Tyndall} - studied glacier motion 
        	\blue{Ignaz Venetz} – suggested the existence of past ice ages
   %Current Glaciers and Facts
		\mysection{Current Glacier Records:} 
		    \mysub{Top Five Longest Non-Polar}
		    \mysubsub {Fedchenko Glacier} in Tajikistan at 77 km
		    \mysubsub {Siachen Glacier}, in the Karakorum range, border between India and Pakistan - 76 km
		    \mysubsub {Biafo Glacier} in Pakistan also by the border - 67 km
		    \mysubsub {Bruggen Glacier} in Chile - 66 km
		    \mysubsub {Baltoro Glacier} in Pakistan at the border - 63 km. \mysub{Longest per continent: } 
		    \mysubsub {Lambert Glacier(Biggest in the world)} in Antarctica(320 mi long, 40 mi wide)
		    \mysubsub {Heard Island Glacier} in Australia(which cover 67 percent of heard island proper)
		    \mysubsub {Siachen Glacier} in Asia with 3 trillion cubic tons of ice
		    \mysubsub {Kilimanjaro's glaciers} in Africa(which are retreating alarmingly)
		    \mysubsub {Vatnojokull Glacier} of Europe (Iceland --> covers 8 percent)
		    \mysubsub {Perito Moreno Glacier} in S.A. which is thriving despite trend of retreat in the globe
		    \mysubsub {Hubbard Glacier} in N.A. (largest tidewater glacier my far). 
		    \mysubsub{Fastest Surge: } The Kutiah Glacier in Pakistan has the record for the fastest glacier surge. In 1953, it moved more than 12km in three months.
		    \mysubsub{Largest Glacier: } The largest glacier in the world is the Lambert-Fisher Glacier in Antarctica. At 400 kilometers (250 miles) long, and up to 100 kilometers (60 miles) wide, this ice stream alone drains about 8\% of the Antarctic Ice Sheet.
		    \mysubsub{LGM} During the maximum point of the last ice age, glaciers covered about 32\% of the total land area.
		    \ddd The word “glacier” comes from the French language and the name is derived from the Latin word glacies meaning “ice“.
		    \mysubsub{Death: } In 1985, a volcano in Colombia that was covered in glaciers erupted, instantly melting the glaciers. Two hours later, a 100 ft deep flood of rock and water traveling 39 feet per second leveled an entire nearby village, killing 20,000 out of its 29,000 residents
		    \mysubsub{Greenland rising: }  Greenland is rising 1 inch per year as glaciers melt, alleviating the land of its weight. 
		    \mysubsub{Europe Glaciers} found in the Alps, Caucasus and the Scandinavian Mountains and Iceland. Most of Europe's large glaciers are in Norway, with the exception of the biggest, which is in Iceland, called the Vatnojokull Glacier.
		    \mysubsub{N.A. Glaciers} Glaciers are in 9 of America's states, in Mexico and of course in Canada. Southernmost in the states is the Lilliput in California. Glaciers in Mexico are in the Pico de Orizaba (Citlaltépetl), Popocatépetl and Iztaccíhuatl, the three tallest mountains in the country.
		    \mysubsub{S.A. Glaciers} S.A. glacier exclusively on the Andes. Apart from this there is a wide range of latitudes on which glaciers develop from 5000 m in the Altiplano mountains and volcanoes to reaching sea level as San Rafael Lagoon ($ 45^\circ $ S) and southwards. South America hosts two large ice fields, the Northern and Southern Patagonian Ice Fields.
		    \mysubsub{Oceania Glaciers} No glaciers remain on the Australia mainland or Tasmania.Heard Island glaciers are located in the territory of Heard Island and McDonald Islands. New Guinea has the Puncak Jaya glacier. New Zealand contains many glaciers,  located near the Main Divide of the Southern Alps in the South Island. They are classed as mid-latitude mountain glaciers. There are eighteen small glaciers in the North Island on Mount Ruapehu.
		    \mysubsub{Africa Glaciers} Only all-season glaciers exsist on Kilimanjaro, Mount Kenya, and the Rwenzori, but seasonally occur in the Drakensberg Range of South Africa, the Stormberg Mountains, and the Atlas Mountains in Morocco.
		    \mysubsub{Antartican Glaciers} Has many outlet glaciers, valley glaciers, cirque glaciers, tidewater glaciers and ice streams e.g. Pine Island Glacier. 
	\mysection{Quick Facts: }
		    \mysubsub{Fresh Water} has 69 percent of the world's supply in glaciers
		    \mysubsub{Number} of glaciers in Alaska is over 100,000
		    \mysubsub{Glacier} and ice sheet all melted = a sea level rise of over 300 feet
		    \mysubsub{Speed} of glaciers is as high as moving 150 feet per day
		    \mysubsub{A single} glacier ice crystal can grow to the size of a baseball
		    \ddd Glaciers most often occur at high latitudes and elevations
		    \ddd Glacier ice is blue because when the snow transitions to firn and glacier ice, \ddd the air bubbles are squeezed out and the natural color shows
		    \ddd Glacier ice is formed by the snow not melting and slowly compressing over the years
		    \ddd The equilibrium line is also called the firn limit 
		    \ddd The order of glacial ice formation is as follows - Snow > granular ice > firn ice > glacial ice
	\mysection{Radom Answers}
	\ddd Glacial movement by internal flow is not very fast. 
	\ddd Ice in the upper central part of a valley glacier moves faster than ice at the sides.
	\ddd Strain rate for ice creep is not usually uniform.
	\ddd Snow from the previous season can recrystallize and form firn ice.
	\ddd The superimposed ice zone is usually near the snow line.
	\ddd If glacial ice starts to internally deform, plastic flow will start to dominate.
	\ddd The peaks of a nunatak are usually free of glacial ice.
	\ddd About half of all insolation received by the upper atmosphere is absorbed by land and oceans, while the remaining half is either reflected or absorbed by the lower atmosphere.
	\ddd If all the sea ice in the Arctic Ocean were to melt, the average sea level will not change.
	\ddd Increased orbital eccentricity will increase seasonal contrast.
	\ddd The Last Glacial Maximum lasted from around 25,000-15,000 years ago.
	\ddd Meltwater from the Laurentide Ice Sheet formed the massive Lake Agassiz.
	\ddd \blue{Rapid, quasi-periodic climatic fluctuations} occurring about 25 times during the last glacial period = \blue{Dansgaard-Oeschger events}. 
	\ddd Two general methods by which glacier can flow = Basal Sliding and Internal Flow. 