\mysection{Glacial Formation}
		\mysub {Glacial Ice:}
		Glacier formation - snow in same area year long and accumulates into masses of ice; ; after the first winter this is known as \vocab{neve}; after two winters (one melt season), snow turns into \vocab{firn}; the ice increases in density over years
		\mysub{Ice Crystal Structure:} 
		Commonly takes the shape of sheets or planes of oxygen atoms joined in a series of open hexagonal rings; ice can form 18 different crystalline phases; tacked in a laminar structure that occasionally deforms by gliding; When this gliding deformation occurs, the bonds between the layers break, and the hydrogen atoms involved in those bonds must become attached to different oxygen atoms
		\mysub{Properties of Ice:} The albedo is 0.5 to 0.9 for snow, 0.3 to 0.65 for firn, and 0.15 to 0.35 for glacier ice
		\mysubsub{Albedo:} lowers the melting point of the glacier due to hydrostatic pressure, where deeper parts of the glacier are colder
		\mysub{Moraine: }
		Material a glacier picks up or pushes as it moves forms moraines along the surface and sides of the glacier. As a glacier retreats, the ice literally melts away from underneath the moraines, so they leave long, narrow ridges that show where the glacier used to be. Glaciers do not always leave moraines behind, however, because sometimes the glacier’s own meltwater washes the material away.
		\mysub{Drumlins: } long, tear-drop-shaped sedimentary formations. What caused drumlins to form is poorly understood, but scientists believe that they were created subglacially as the ice sheets moved across the landscape during the various ice ages. Theories suggest that drumlins might have been formed as glaciers scraped up sediment from the underlying ground surface, or from erosion or deposition of sediment by glacial meltwater, or some combination of these processes. Because the till, sand, and gravel that form drumlins are deposited and shaped by glacier movement, all drumlins created by a particular glacier face the same direction, running parallel to the glacier's flow. Often, hundreds to thousands of drumlins are found in one place, looking very much like whale backs when seen from above
		\mysub{Structure: }The region near the head of the glacier where snow is converted to firn and then ice is called the zone of accumulation.  The region near the foot of the glacier is called the zone of ablation; this is where ice is lost by melting, evaporation or calving (to make icebergs).  Separating the accumulation zone from the ablation zone is the equilibrium line. The equilibrium line is located at the equilibrium line altitude (ELA).
	%Mass Balance and Flow
	\mysection{Mass-balance and Flow}
		\mysub{Ablation and Accumulation Zones:}
		\mysubsub{Accumulation Zones} are the part of the glacier that has more accumulation than ablation
		\mysubsub{Ablation Zones} are the part of the glacier that has more ablation than accumulation, and the calculate mass balance, just add the two figures
		\mysub{Equilibrium Lines} The point across a glacier where accumulation is equal to ablation; The lower the altitude of this line, the faster the glacier advances, and vice versa
		\mysub{Relation of flow to elevation and gradient:}
		\mysubsub{Gradient -} the higher the gradient, the faster the floor of the glacier; Glaciers steeper in maritime climates and temperate latitudes then in continental climates and polar latitudes.
		\mysubsub{Elevation -}Lower the altitude, the less likely it is to find a glacier; So, the lower altitude causes glaciers to be valley or piedmont glaciers rather then cirque, and more active.
    %Glacier Types and Forms
	\mysection{Glacier Types and Forms}
		\mysub{Ice Caps} have an area less than 50,000 square km
		\mysub{Ice Sheets} have an area greater than this; only extant ice sheets are in Antarctica and Greenland
		\mysub{Alpine Glaciers} Begin high in the mountains from cirques and then valley, then piedmont
		\mysubsub{Cirque Glaciers} glacial ice collected in bowl shaped depression high in the mountains 
		\mysubsub{Hanging Glaciers}  A hanging glacier originates high on the wall of a glacial valley and descends only part of the way to the surface of the main glacier and abruptly stops, typically at a cliff.
		\mysubsub{Piedmont Glaciers} valley glaciers that spread out onto a flat lowland
		\mysubsub{Tidewater Glaciers} glaciers that flow to the sea
		\mysubsub{Valley Glaciers} a thin stream of ice that takes up a valley and originates from one or many cirques
		\mysubsub{Apron Glaciers} Glaciers that cling to steep mountainsides and are very avalanche prone.
	%Glacial Features
	\mysection{Glacial Features}  
	    \mysub{Ice Stream} section of fast flow within a glacier that make up most o the way that a glacier discharges ice and sediment.
		\mysub{Ice Shelves} a suspended section of ice connected to a landmass that forms when a glacier flows down to the ocean’s surface.
		\mysub{Ice Rise} an obvious dome shaped bump in the ice of a glacier formed when the seabed under a glacier has a similar bump, located with valley glaciers.
		\mysub{Ice Stream} a long, narrow sheet of ice that extends out over the ocean that forms when a valley glacier moves very rapidly onto the ocean.
		\mysub{Nunatak} an exposed, often rocky element of a ridge, mountain, or peak not covered with ice or snow within an ice field or glacier; also called glacial islands.
		\mysub{Crevasses} deep cracks in glacier ice caused by the stress of the ice moving over rocky terrain underneath, indicate that glacier is under different types of stress as it flows. If crevasses close up, it shows that a glacier is flowing over an area of less gradient. 
		\mysub{Ogives} alternating bands of light and dark ice that forms ridges arcs of ice bending downstream. This shows that a glacier is moving faster in the center, creating these arched bands, or is moving over steeper terrain.
		\mysub{Ice Falls} glaciers flow over an steep drop or squeeze through an narrow place characterized by rapid flow and a crevassed surface. Happens when a glacier flows over a steep surface or narrows.
	%Hydrology
	\mysection{Hydrology}
		\ddd Glacier hydrology is the study of the flow of water through glaciers
		\ddd Glacier ice is permeable, with a network of microscopic veins and lenses of water.
		\mysub{Glacier Permeability} 
			\ddd The rate at which water percolates through the glacier is dependent on salinity, pressure and temperature. 
			\ddd  The rate at which ice seeps through the ice, however, is so slow, that for practical reasons ice can generally be considered impermeable
		\mysub{Supraglacial (surface) water} on a glacier is formed by the ice melting during the summer.
		\mysub{Ablation}: Surface melt; occurs in hard packed snow (firn: the transistional state between snow and ice)
		\mysub{Swamp zone} If a firn becomes saturated all the way to the surface it becomes a ‘swamp zone’; Swamp zone moved up glacier as the melt season progresses.
		\ddd Much of the meltwater runoff in Antarctica is restricted to coastal areas and ice shelves during the summer seasons
		\mysub{Englacial Hydrology}
			\ddd \vocab{Moulins} are vertical shafts cut by the water.
			\ddd Water cascades down these into the ice sheet. Despite the pressures within the ice sheet, moulins remain open by constant melting by the water
		\mysub{Subglacial Hydrology}
			\ddd Basal meltwater flowing through large subglacial networks impact glacial erosion and ice velocity. 
		\mysub{Proglacial Drainage}
			\ddd Abundant meltwater can form large braided river plains, or sandur
			\ddd Runoff is less in Antarctica, and meltwater in the northern Antarctic Peninsula tends to be restricted to small braided streams
			\ddd These streams redeposit glacial sediments and rework glacial landforms
	\mysection{Laurentide Ice Sheet}
		\ddd The mass of ice in the Greenland Ice Sheet has begun to decline. From 1979 to 2006, summer melt on the ice sheet increased by 30 percent, reaching a new record in 2007
		\ddd Antartica has not shown noticeable changes. 
		\ddd Antartic peninsula has seen changes which is the part that sticks out of the continent
	