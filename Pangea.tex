\mysection{Formation of Pangea}
The forming of supercontinents and their breaking up appears to have been cyclical through Earth's history. There may have been several others before Pangaea. The fourth-last supercontinent, called Columbia or Nuna, appears to have assembled in the period 2.0–1.8 Ga.[15][16] Columbia/Nuna broke up and the next supercontinent, Rodinia, formed from the accretion and assembly of its fragments. Rodinia lasted from about 1.1 billion years ago (Ga) until about 750 million years ago, but its exact configuration and geodynamic history are not nearly as well understood as those of the later supercontinents, Pannotia and Pangaea.
When Rodinia broke up, it split into three pieces: the supercontinent of Proto-Laurasia, the supercontinent of Proto-Gondwana, and the smaller Congo craton. Proto-Laurasia and Proto-Gondwana were separated by the Proto-Tethys Ocean. Next Proto-Laurasia itself split apart to form the continents of Laurentia, Siberia and Baltica. Baltica moved to the east of Laurentia, and Siberia moved northeast of Laurentia. The splitting also created two new oceans, the Iapetus Ocean and Paleoasian Ocean. Most of the above masses coalesced again to form the relatively short-lived supercontinent of Pannotia. This supercontinent included large amounts of land near the poles and, near the equator, only a relatively small strip connecting the polar masses. Pannotia lasted until 540 Ma, near the beginning of the Cambrian period and then broke up, giving rise to the continents of Laurentia, Baltica, and the southern supercontinent of Gondwana.
In the Cambrian period, the continent of Laurentia, which would later become North America, sat on the equator, with three bordering oceans: the Panthalassic Ocean to the north and west, the Iapetus Ocean to the south and the Khanty Ocean to the east. In the Earliest Ordovician, around 480 Ma, the microcontinent of Avalonia – a landmass incorporating fragments of what would become eastern Newfoundland, the southern British Isles, and parts of Belgium, northern France, Nova Scotia, New England, South Iberia and northwest Africa – broke free from Gondwana and began its journey to Laurentia.[17] Baltica, Laurentia, and Avalonia all came together by the end of the Ordovician to form a minor supercontinent called Euramerica or Laurussia, closing the Iapetus Ocean. The collision also resulted in the formation of the northern Appalachians. Siberia sat near Euramerica, with the Khanty Ocean between the two continents. While all this was happening, Gondwana drifted slowly towards the South Pole. This was the first step of the formation of Pangaea.[18]
The second step in the formation of Pangaea was the collision of Gondwana with Euramerica. By the Silurian, 440 Ma, Baltica had already collided with Laurentia, forming Euramerica. Avalonia had not yet collided with Laurentia, but as Avalonia inched towards Laurentia, the seaway between them, a remnant of the Iapetus Ocean, was slowly shrinking. Meanwhile, southern Europe broke off from Gondwana and began to move towards Euramerica across the newly formed Rheic Ocean. It collided with southern Baltica in the Devonian, though this microcontinent was an underwater plate. The Iapetus Ocean's sister ocean, the Khanty Ocean, shrank as an island arc from Siberia collided with eastern Baltica (now part of Euramerica). Behind this island arc was a new ocean, the Ural Ocean.
By the late Silurian, North and South China split from Gondwana and started to head northward, shrinking the Proto-Tethys Ocean in their path and opening the new Paleo-Tethys Ocean to their south. In the Devonian Period, Gondwana itself headed towards Euramerica, causing the Rheic Ocean to shrink. In the Early Carboniferous, northwest Africa had touched the southeastern coast of Euramerica, creating the southern portion of the Appalachian Mountains, the Meseta Mountains and the Mauritanide Mountains. South America moved northward to southern Euramerica, while the eastern portion of Gondwana (India, Antarctica and Australia) headed toward the South Pole from the equator. North and South China were on independent continents. The Kazakhstania microcontinent had collided with Siberia. (Siberia had been a separate continent for millions of years since the deformation of the supercontinent Pannotia in the Middle Carboniferous.)
Western Kazakhstania collided with Baltica in the Late Carboniferous, closing the Ural Ocean between them and the western Proto-Tethys in them (Uralian orogeny), causing the formation of not only the Ural Mountains but also the supercontinent of Laurasia. This was the last step of the formation of Pangaea. Meanwhile, South America had collided with southern Laurentia, closing the Rheic Ocean and forming the southernmost part of the Appalachians and Ouachita Mountains. By this time, Gondwana was positioned near the South Pole and glaciers were forming in Antarctica, India, Australia, southern Africa and South America. The North China block collided with Siberia by the Late Carboniferous, completely closing the Proto-Tethys Ocean.
By the early Permian, the Cimmerian plate split from Gondwana and headed towards Laurasia, thus closing the Paleo-Tethys Ocean, but forming a new ocean, the Tethys Ocean, in its southern end. Most of the landmasses were all in one. By the Triassic Period, Pangaea rotated a little and the Cimmerian plate was still travelling across the shrinking Paleo-Tethys, until the Middle Jurassic. The Paleo-Tethys had closed from west to east, creating the Cimmerian Orogeny. Pangaea, which looked like a C, with the new Tethys Ocean inside the C, had rifted by the Middle Jurassic, and its deformation is explained below.
