\mysection{Glacial Basics}
	\mysub{Aerosol:} a colloidal suspension of particles dispersed in air or gas.
	\mysub{Firn:} the intermediate state between snow and glacial ice
	\mysub{Accumulation:} Accumulation is when glaciers gain more mass through snowfall, windblown snow, avalanches, rime ice (freezing water vapor), refreezing meltwater.
	\mysub{Ablation:} Ablation is when glaciers loose mass through surface melt, surface meltwater runoff, sublimation, avalanching and windblown snow.
\mysection{Glacier Movement}
	\mysub{acceleration: }  caused by thinning of ice sheet due to ablation, increased heat form geothermal gradient, lubrication of basal sliding surface via meltwater. 
	\mysub{Slow down: } caused by ice sheet thickens, freezes to basal bed surface, a reduction in the thermal gradient, change in elevation and slope. 
	\mysub{Ice sheet mechanism for ice flow: } will transport ice via ice streams, where ice flows from central accumulation zones into fast flowing ice streams, which carve out deep channels-deeper, increased warmth, faster flow. Ex: Lambert Glacier, Antarctica.
	\mysub{Modern ice caps} found mostly in polar and subpolar regions (Greenland, Antarctica, North Pole, Himlayas) 
	\mysub{Grounding line: } point at which ice sheet begins to float. 
	\mysub{Instability of ice shelves: } Indicates that ice shelves are fairly unstable, and may have feedback mechanisms associated with them- ie ice collapses, the grounding line moves out (since shelves are not attached to the base of anything) and then increases the drawdown of the ice sheet- allowing further grounding line retreat.
\mysection{Glacial Landforms}
	\mysub{Medial Moraines: } tell you two glaciers coexisted at same time and neither was higher than height of medial moraine. If they ask what direction you need drumlins!!! Order of how much underlying topography exerts influence: ice sheet, ice cap, ice field, valley glacier. You see asymmetrical profiles bc ice flow builds up sediment and material and on uphill side. Glacial retreat influences glacial sediment deposition by depositing fine grained sediment and material. It forms kames, eskers, and tors. \ddd \textbf{Increased melting} -> glacier moves faster from water lubrication. Liquid water found at bases from high pressure creating compression melting. Subglacial channels, makes, eskers, drumlins are evidence of water at base of glacier. Why are glaciers hazardous? They can dam lakes, rapidly melt, and release flood water. Subglacial lakes can generate large floods too.
	\mysub{Atmosphere redistributes heat} over surface of earth through thermal convection – hot air rises and then sinks to the poles. Sea level drops during ice age, to check this you can investigate changes in oxygen isotope records (reflective of changes in global ice volume), or look for transgressive or regressive sedimentary sequences. \
	\mysub{Why are there icebergs on coast of Newfoundland}, but not Norway(much more North)? The Norwegian current brings warm water up the coast of Europe, which warms the eastern side of the ocean basin. Newfoundland is adjacent to the source of formation for the North Atlantic Deep Water (NADW), and is cooled by air and water circulating down from the attic. 
	\vocab{Oceanic sediment core –} \ddd Core w/ primarily fine grained muds, but every ten cm you see large 5cm clasts of a dark mafic rock. These are \textbf{drop stones}, eroded out under a glacier or ice sheet, which calved icebergs into the ocean, which drifted to the location of the core and melted.
	\blue{Epishelf:} a floating ice shelf (the seaward extension of glaciers that terminate in the ocean) may block the mouth of a fjord, creating a unique type of lake called an “Epishelf Lake”.
	Densities - New snow: 0.05 to 0.07 g/cm³ Damp new snow: 0.1 to 0.2 g/cm³ Settled snow: 0.2 to 0.3 g/cm³ Depth hoar: 0.1 to 0.3 g/cm³ Wind packed snow: 0.35 to 0.4 g/cm³ Firn: 0.4 to 0.83 g/cm³ Very wet snow and firn: 0.7 to 0.8 g/cm³ Glacier ice: 0.83 to 0.92 g/cm³
%Global Connections  
\mysection{Global Connections}
	\mysub{Glacial Buzzsaw hypothesis: }
		a hypothesis claiming erosion by warm-based glaciers is key to limit the height of mountains above certain threshold altitude.To this the hypothesis adds that great mountain massifs are leveled towards the equilibrium line altitude (ELA), which would act as a “climatic base level”Starting from the hypothesis it has been predicted that local climate restricts the maximum height that mountain massifs can attain by effect of uplifting tectonic forces
	\mysub{Plucking} 
		is glacial phenomenon that is responsible for the erosion and transportation of individual pieces of bedrock, and this results in a eroded headwall in most valley glaciers. In addition, this is a method that rocks are transported through a glacier
	\mysub{Thermohaline Circulation: }
		Is a part of the large-scale ocean circulation that is driven by global density gradients created by surface heat and freshwater fluxes. The biggest determiners are \blue{temperature} and \blue{salinity}.
	\mysub{Temperature-Salinity}
		As the Earth continues to warm and Arctic sea ice melts, the influx of freshwater from the melting ice is making seawater at high latitudes less salty and hence less dense.
	\mysub{Cooling of Europe!}
		Cooling in Western Europe! Currently the ocean currents carry warmth from the tropics up to the high latitudes. That warmth is lost to the atmosphere keeping the temperatures of places like England, Norway, and many other counties in northern Europe a bit milder than other places at the same latitude. If the Global Ocean Conveyor were to stop completely, the average temperature of Northern Europe would cool 5° to 10° Celsius, but even a slow down could lead to a measurable cooling.
	\mysub{North Atlantic Deep Water (NADW)}
		North Atlantic Deep Water (NADW) is a deep water mass formed in the North Atlantic Ocean. Thermohaline circulation (properly described as meridional overturning circulation) of the world's oceans involves the flow of warm surface waters from the southern hemisphere into the North Atlantic. It is actually a nutrient minima because it is formed from nutrient depleted surface layers that sunk.
	\mysub{Ozone Holes}
		Ozone holes form in the poles because they are surrounded by oceans which cause a whirlpool effect with CFC concentrated water and air. 
	\mysub{Paleoclimate Proxy Information}
		\ddd \vocab{climate proxies} are preserved physical characteristics of the past that stand in for direct meteorological measurements[2] and enable scientists to reconstruct the climatic conditions over a longer fraction of the Earth's history.
		\ddd Reliable global records of climate only began in 1880s
		\mysubsub{Examples of Proxies: }
			include ice cores, tree rings, sub-fossil pollen, boreholes, corals, lake and ocean sediments, and carbonate speleothems. The character of deposition or rate of growth of the proxies' material has been influenced by the climatic conditions of the time in which they were laid down or grew. Chemical traces produced by climatic changes, such as quantities of particular isotopes, can be recovered from proxies. Some proxies, such as gas bubbles trapped in ice, enable traces of the ancient atmosphere to be recovered and measured directly to provide a history of fluctuations in the composition of the Earth's atmosphere.
	\mysub{Atmosphere}
	Past 200 yrs: Past 200 yr: CO2 went up by 40\% and Methane by 200\% - 300\%; Glaciers Reflect heat from the sun; increased dust and soot from grazing, farming, and burning of fossil fuels and forests, are also causing glacier retreat by
	\ddd Past 200 yr: CO2 went up by 40\% and Methane by 200\% - 300\%, which glaciers have the ability to combat
	\ddd Reflect heat from the sun, 
	\ddd increased dust and soot from grazing, farming, and burning of fossil fuels and forests, are also causing glacier retreat (albedo)
	\ddd layers of dust and soot are darkening the color of glaciers and snowpacks, causing them to absorb more solar heat and melt more quickly, and earlier in spring.
	\ddd \blue{Albedo}, or "whiteness," is a scientific term meaning reflectivity
	\ddd Cooking stoves (biomass stoves) darken snow and ice in mountainous regions. In The himalayas this is bad because the Yangtze, Yellow, Mekong, and Ganges rivers all feed from glaciers
	\ddd 90\% of Himalayan Glacier Melting Caused by Aerosols and Black Carbon
	\ddd \vocab{Aerosol}: a colloidal suspension of particles dispersed in air or gas.
	\textbf{reducing albedo}.
	\mysub{Ocean} If glacier melted 
	\mysubsub{sea level would rise by: } All of Greenland (7.2m); West Antarctic Ice Sheet (3.2m). All of Antarctica (57m). 
	\ddd seal level has risen by 4 to 8 inches over the past century
	\ddd rate of rise over the past 20 years has been 0.13 inches (3.2 millimeters) a year
	\mysub{Lithosphere} When glaciers erode the rock underneath them, they release carbon gases trapped in the lithosphere. Also, when ice sheets weigh down on the sea floor, the cause depression in the earth's lithosphere, and the edges are called fore bulges, which are massive hills that areas like America's east coast lie upon. When these sink, the depressions left rise, causing a reshuffling of the earth's lithosphere. This is called glacial isostatic adjustment.
	\vocab{basal sliding}. when the ice slides over the land with a layer of water acting as a lubricant and reducing the friction between land and ice. pressure from the weight of the ice reduces the melting point at the base of the glacier which allows the ice to melt, allowing water to be present.  glaciers can move in even the coldest of climates.