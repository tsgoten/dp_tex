\mysection{Recent Records of Cryospheric Change}
		\vocab{Cryosphere}: portions of Earth’s surface where water is in solid form including sea ice, lake ice, river ice, snow cover, glaciers, ice caps, ice sheets, and frozen ground (which includes permafrost).
		\mysub{Larsen Ice Shelf}
			a long ice shelf in the northwest part of the Weddell Sea, extending along the east coast of the Antarctic Peninsula from Cape Longing to Smith Peninsula
			\ddd The collapse of Larsen B has revealed a thriving chemotrophic ecosystem 800 m (half a mile) below the sea.
			\ddd 31 January 2002 - March 2002 Larsen B sector partially collapsed and parts broke up. 
			\ddd Larsen B was stable for 10,000 years, but due to warm currents eating away the underside of the shelf it collapsed. 
			\ddd 3,250-square-kilometer (1,255-square-mile) section collapsed (size of Rhode island)
		\mysub{Kilimanjaro}
			\ddd Kilimanjaro's shrinking northern glaciers, thought to be 10,000 years old, could disappear by 2030
			\ddd The northern ice field, which holds most of the remaining glacial ice, lost more than 140 million cubic feet of ice in the past 13 years
			\ddd Approximately 29\% of the volume and 32\% of the surface area of the ice sheet has been lost since 2000.
			\ddd No real reason is known, with possible links to global warming and less snowfall
		\mysub{Amundsen Sea Embayment}
			\ddd is located off of west Antarctica and the ice that drains into it is roughly 3 km thick
			\ddd Recently, this sheet has significantly thinned because of shifts in wind patterns that allow warmer water to flow under the ice, and is already melting enough to raise the global sea level by 0.2 mm per year. 
			\ddd Two of Antarctica's largest glaciers drain into this basin and if they were to melt, the sea level could increase by up to 3 yards. 
			\ddd The weak underbelly of the West Antarctic Ice Sheet, and if it were to collapse, could destabilize the entire west antarctic ice sheet
	\mysection{Post-Glacial Landscape}  
		\mysub{Erosional Features:}
		\mysubsub{Cirques} a bowl shaped basin formed when a glacier erodes under the bergschrund(a crevasse at or near the head of a glacier) which opens in the early summer, exposing the rock underneath to frost action and causes upper rock to avalanche and scour the floor beneath in the bowl shape, or the bowl left behind from a cirque glacier.
		\mysubsub{Tor} a free-standing rock outcropping that abruptly rises from the surrounding environment, formed at first by erosion and weathering of the ground surrounding it.
		\mysubsub{U-Shaped Valley} happen when valley glaciers advance, eroded a u-shaped depression in the land, and then recede, leaving this U-shaped valleys and mountains behind.
		\mysubsub{Hanging Valley} as a smaller glacier at a higher elevation joins a lower, but larger valley glacier, and they recede, the u shaped valley created by the smaller glaciers opens up onto the lower depression formed by the larger glacier.
		\mysubsub{Aretes} a sharp, crested ridge that separates the heads of two opposing cirques where glaciers used to reside and carved this thin ridge.
		\mysubsub{Horns}when glaciers erode three or more aretes, ending with sharp, vertical peak.
		\mysubsub{Stritations/Grooves} are carved into bedrock as glaciers pass over it.
		\mysubsub{Rôche moutonnée} occurs when a glacier claws itself up a hill, it damages the surface, leaving jagged and irregular on that side, but as it slides down, it polishes the surface, leaving the other side of the same rock smooth and even.
		\mysubsub{Tarn} a lake left in a bowl shaped depression by a receding cirque glacier.
		\mysub{Depositional Features: }
		\mysubsub{Moraines} are rocks or sediment deposited by a glacier, typically at its edges.
		\mysubsubsub{End/Terminal} a moraine that forms at the leading edge of a glacier marking its furthest advance, formed by debris pushed to the front of a glacier.
		\mysubsubsub{Recessional} a series of ridges formed parallel to the terminal moraine and form when a glacier temporarily stops receding.
		\mysubsubsub{Lateral} a series of parallel ridges deposited along the sides of a glacier that form when frost shatters the valley walls and causes them to collapse.
		\mysubsubsub{Medial}  a ridge of a moraine that forms in the center of a valley. It forms when two glaciers meet and the debris on the edges of the adjacent valley sides join and are carried on top of the enlarged glacier.
		\mysubsubsub{Ground} an irregular blanket of sediment most often deposited by continental glaciers
		\mysubsub{Kettles} when a block of ice calves and is submerged into sediment, and subsequently melts, the hole it leaves behind is called a kettle.
		\mysubsub{Kames} a hill of sand, sediment and till that forms on top of a retreating glacier then is deposited on the land underneath as the glacier further melts.
		\mysubsub{Drumlins} an elongated hill shaped like a inverted spoon aligned with the ice flow that forms under the glacier bed and a left when the glacier retreats.
		\mysubsub{Eskers} a long ridge composed of sediment and gravel formed under a glacier when subglacial rivers in ice walled tunnels left sediment underneath then and when the retaining walls of ice melted away
		\mysubsub{Erratics} pieces of rocks that are foreign to their surroundings regarding their size and type. They are transported by glaciers for thousands of miles.
		\mysubsub{Moulins} are vertical shafts created in a glacier by waater within it.