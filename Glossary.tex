
\pagebreak
\vocab{Ablation Hollows: } Depressions in the snow surface caused by the sun or warm, gusty wind
\vocab{Ablation Moraine: } Mound or layer of moraine in the ablation zone of a glacier; the rock has been plucked from the mountainside by the moving glacier and is melting out on the ice surface
\vocab{Ablation Season: } Period during which glaciers lose more mass than they gain; usually coincides with summer
\vocab{Ablation Zone: } Area or zone of a glacier where snow and ice ablation exceed accumulation
\vocab{Abrasian: } rocks within the ice acting like sandpaper to smooth and polish the surface below; pulverized rock produced is called rock flour; glacial striations: ice at the bottom of a glacier contains large rock fragments, and long scratches and grooves; give clues to direction of travel
\vocab{Accommodation Space Equation: } represents a simple volume balance, with the terms on the left controlling the amount of space that can be occupied by sediments and water and the terms on the right describing how much water or sediment fills the accommodation space
\vocab{Accumulation Area: } Area of a glacier where more mass is gained than lost
\vocab{Accumulation Season: } Period during which a glacier gains more mass than it loses usually coincides with winter
\vocab{Accumulation Zone: } Area of a glacier where more mass is gained than lost
\vocab{Advance: } When a mountain glacier's terminus extends farther down valley than before; glacial advance occurs when a glacier flows down valley faster than the rate of ablation at its terminus
\vocab{Advancing glacier: } When accumulation exceeds ablation and the glacier advances
\vocab{Albatross: } the largest sea bird in Antarctica
\vocab{Alluvium: } Sediment eroded from adjacent areas and deposited by running water in and along rivers and streams. sorted.
\vocab{Alpine Glacier: } A glacier that is confined by surrounding mountain terrain; also called a mountain glacier
\vocab{Altithermal: } A period of time in the mid-Holocene when climate was generally warmer. Also called the hypsithermal and/or climatic optimum
\vocab{Amundsen and Scott: } 2 explorers that raced to be the first to the South Pole
\vocab{Antarctica: } the coldest, highest, and driest continent
\vocab{Antarctic Peninsula: } the peninsula that juts out on the northwestern part of Antarctica
\vocab{Arête: } Sharp, narrow ridge formed as a result of glacial erosion from both sides
\vocab{Avalanche: } A great mass of ice, earth, or snow mixed with rocks sliding down a mountain
\vocab{Band Ogives: } Alternate bands of light and dark on a glacier; usually found below steep narrow icefalls and thought to be the result of different flow and ablation rates between summer and winter
\vocab{Basal Sliding: } The sliding of a glacier over bedrock; melting point of ice decreases with pressure.
\vocab{Bergschrund: } (Rimaye) Crevasse that separates flowing ice from stagnant ice at the head of a glacier
\vocab{bergy bits: } A large chunk of glacial ice (or a very small iceberg) which floats in the sea.
\vocab{bottom bergs: } Icebergs which originate from near the base of a glacier. They are usually black from trapped rock material or dark blue because of old, coarse, bubble-free ice and sit low in the water due to the weight of embedded rocks.
\vocab{Branched-Valley Glacier: } Glacier that has one or more tributary glaciers that flow into it; distinguished from a simple valley glacier that has only a single tributary glacier
\vocab{Brittle Zone: } The upper 50 meters of a glacier that breaks as the ice moves
\vocab{Bubble Rock: } Name of this one erratic in Acadia national park.
\vocab{Budget of Glacier: } as terminus, or bottom of glacier, retreats, zone of wastage decreases$ \rightarrow $new balance will be reached eventually between accumulation and wastage, and ice front will become stationary; no matter how margin is moving ice within the glacier continues to flow forward; even if glacier is retreating, but not enough to stop ablation
\vocab{Buttressing: } This is due to "iceshelf buttressing," the term used to describe the resistive forces imparted to the ice sheet by its ice shelves
\vocab{Calving: } process by which a block of a glacier breaks off and falls into the sea to form an iceberg
\vocab{Catchment Glacier: } A semi permanent mass of firn formed by drifted snow behind obstructions or in the ground; also called a snowdrift glacier or a drift glacier
\vocab{Chanelled Scablands: } Land after a giant flood.
\vocab{Chattermarks: } Striations or marks left on the surface of exposed bedrock caused by the advance and retreat of glacier ice
\vocab{Chattermarks: } Striations or marks left on the postglacial exposed bedrock caused by the striking of englacial debris against the bedrock surface during glacial movement.
\vocab{Cirque: } a bowl-shaped depression at the head of a glacial valley
\vocab{Cirque: } Bowl shaped or amphitheater usually sculpted out of the mountain terrain by a cirque glacier
\vocab{Cirque Glacier: } Glacier that resides in basins or amphitheaters near ridge crests; most cirque glaciers have a characteristic circular shape, with their width as wide or wider than their length
\vocab{Clay: } Plate minerals with water
\vocab{Col: } Ridge between 2 cirques
\vocab{Col: } the lowest point of a ridge or saddle between two peaks, typically affording a pass from one side of a mountain range to another.
\vocab{Cold Glacier: } Glacier in which most of the ice is below the pressure melting point; nonetheless the glacier's surface may be susceptible to melt due to incoming solar radiation, and the ice at the rock/ice interface may be warmed as a result of the natural (geothermal) heat from the earth's surface
\vocab{Compression Flow: } Flow that occurs when glacier motion is decelerating down-slope
\vocab{Constructive Metamorphism: } Snow metamorphism that adds molecules to sharpen the corners and edges of an ice crystal
\vocab{Continental Glacier: } A glacier that covers much of a continent or large island
\vocab{Continental glaciers: } Continents covered in ice (Antarctica and Greenland)
\vocab{Cordilleran Ice Sheet: } The ice cap that covered much of the mountains in the northwestern part of North America during the Pleistocene Epoch.
\vocab{Corrie: } A hollow containing a small glacier that is armchair shaped
\vocab{Crag and Tail: } Rocky hill followed by a tale of till
\vocab{Crayocinite: } Depression in a glacier formed when a rock on the surface is heated and melts the surrounding snow.
\vocab{Crevasse: } A deep crack or fissure in the ice of a glacier.
\vocab{Crevasse Hoar: } A kind of hoarfrost; ice crystals that develop by sublimation in glacial crevasses and in other cavities with cooled space and calm, still conditions under which water vapor can accumulate; physical origin is similar to depth hoar
\vocab{Cuesta: } A ridge with a gentle slope on one side and a steep slope on the other, often resulting from the movement of a glacier over a rock outcrop. Cuestas are large scale features analogous to rock knobs (roche moutonnée).
\vocab{Cwm: } same thing as a cirque.
\vocab{Dead Ice: } Any part of a glacier which has ceased to flow; dead ice is usually covered with moraine
\vocab{De geer moraine: } underwater moraine; parallel ridges like ripple marks. Only a few 100 ft apart.
\vocab{depth hoar: } In snow, relatively large (1 to several mm diameter), cohesionless, coarse, faceted snow crystals resulting from the presence of steep temperature gradients within the snowpack.
\vocab{Diamicton: } Diamicton is a general term used to describe a non-sorted or poorly sorted, sometimes non-calcareous, terrigenous or marine sediment containing a wide range of particle sizes derived from a broad provenance
\vocab{Dirt Cone: } A cone-shaped formation of ice that is covered by dirt; a dirt cone is caused by a differential pattern of ablation between the dirt-covered surface and bare ice
\vocab{Discharge: } The total volume of ice passing through a specified cross section of the glacier during a particular unit of time/
\vocab{Draft: } The depth below the water level, usually sea level, to whichthe base (or keel) of an iceberg penetrates is called its draft.
\vocab{Drain Channel: } Preferred path for meltwater to flow from the surface through a snow cover
\vocab{Drift: } Any material carried and deposited by a glacier.
\vocab{Drift Glacier: } A semi-permanent mass of firn formed by drifted snow behind obstructions or in the ground; also called a catchment glacier or a snowdrift glacier
\vocab{Dropstone: } A rock that was carried elsewhere by a glacier or iceberg and deposited when the ice melts, the rock sinks to the bottom of the body of water and becomes part of a sedimentary rock.
\vocab{Drumlin: } Remnant elongated hills formed by historical glacial action; it is not clear exactly how they are formed and why they form only in some glaciated regions
\vocab{Drumlin: } Tear drop shaped hill made of reworked glacial moraines
\vocab{Dry Bottom Glacier: } A glacier so cold that its base remains frozen to the substrate, also called a polar glacier. Occur in regions where atmospheric temperatures stay so cold all year long that the glacial ice remains below melting. Mars also has polar glaciers.
\vocab{Dump Moraine: } A mound or layer of moraine formed along the edge of a glacier by rocks that fall off the ice; sometimes called a ground moraine
\vocab{emperor penguin: } the largest of the 6 types of penguins found in Antarctica
\vocab{End Moraine: } An arch-shaped ridge of moraine found near the end of a glacier
\vocab{Englacial: } All the glacial environments which occur within the ice itself are called englacial environments.
\vocab{Equilibrium Line: } the boundary between the zone of accumulation and the zone of ablation
\vocab{Equilibrium Zone: } Zone of a glacier in which the amount of precipitation that falls is equal to the amount that melts the following summer
\vocab{Erratics: } Large pieces of rock that have been transported away from their source areas by moving glacier sheets
\vocab{Esker: } A sinuous ridge of sedimentary material (typically gravel or sand) deposited by streams that cut channels under or through the glacier ice
\vocab{Extending flow: } when glacier motion is accelerating down-slope
\vocab{fabric: } In tills which have been oriented by flowing water, fabric indicates the preferred orientation of the grains. Sedimentologists would refer to this as "imbrication."
\vocab{False ogives: } bands of light and dark on a glacier that were formed by rock avalanching
\vocab{Felsenmeers: } A large area blanketed with angular debris from outcrops which have suffered repeated cycles of freezing and thawing.
\vocab{Fern and Neve: } Snow under pressure. When granules of ice are put under pressure, the granules weld together
\vocab{Fjord: } glacial troughs that fill with seawater
\vocab{flutes: } Long grooves gouged by englacial debris on subglacial pavement parallel to direction of glacial movement.
\vocab{Foliation: } layering in glacier ice that has distinctive crystal sizes and/or bubbles; foliation is usually caused by stress and deformation that a glacier experiences as it flows over complex terrain, but can also originate as a sedimentary feature
\vocab{Forbes bands: } alternate bands of light and dark on a glacier; usually found below steep narrow icefalls and thought to be the result of different flow and ablation rates between summer and winter
\vocab{foredeepening topography: } The ground below an ice sheet may be bowl-shaped with the inner part being deeper than the ground around the edges because glaciers erode preglacial material and subsidence due to the weight of the ice.
\vocab{Forel stripes: } shallow, parallel grooves on the face of a large melting ice crystal
\vocab{Frazil Ice: } Disorganized, slushy ice crystals in the water column, usually near the water surface. Frazil ice is the first stage in the formation of sea ice.
\vocab{Furrow: } Long grooves in subglacial till or pavement gouged by englacial debris. (see flutes)
\vocab{Gargantuan steps: } Giant terraces in a u shaped valley formed by glacial plucking
\vocab{Gendarmes: } Ice towers such as seracs and penitantes.
\vocab{Geyser: } Fountain that develops when water from a conduit is forced up to the surface of a glacier; also called a negative mill
\vocab{Glacial advance: } when a mountain glacier's terminus extends farther downvalley than before; occurs when a glacier flows downvalley faster than the rate of ablation at its terminus
\vocab{Glacial Erratic: } a boulder swept from its place of origin by glacier advance or retreat and deposited elsewhere as the glacier melted; after glacial melt, the boulder might be stranded in a field or forest where no other rocks of its type or size exist
\vocab{Glacial Formation: } 1)Loose snow (90\% air), 2) granular snow (50\% air), 3) firn (25\% air), 4) fine-grained ice (<20\% air), 5) coarse grained ice (<20\% air)
\vocab{Glacial grooves: } grooves or gouges cut into the bedrock by gravel and rocks carried by glacial ice and meltwater; also called glacial striations
\vocab{Glacial Incorporation: } A form of glacial erosion where the ice surrounds debris so the debris starts to move with the ice.
\vocab{Glacial moraine: } Materials carried and deposited by glaciers
\vocab{Glacial Polished Surface: } A polished rock surface created by the glacial abrasion of the underlying substrate.
\vocab{Glacial Rebound: } The process by which the surface of a continent rises back up after an overlying continental ice sheet melts away and the weight of the ice is removed. Takes thousands of years
\vocab{Glacial retreat: } when the position of a mountain glacier's terminus is farther upvalley than before; occurs when a glacier ablates more material at its terminus than it transports into that region
\vocab{Glacial striations: } grooves or gouges cut into the bedrock by gravel and rocks carried by glacial ice and meltwater; also called glacial grooves
\vocab{Glacial Subsidence: } The sinking of the surface of a continent caused by the weight of an overlying glacial ice sheet.
\vocab{Glacial till: } accumulations of unsorted, unstratified mixtures of clay, silt, sand, gravel, and boulders; the usual composition of a moraine
\vocab{Glacial Toe: } The leading edge or margin of a glacier
\vocab{Glacial trough: } a large u-shaped valley formed from a v-shaped valley by glacial erosion
\vocab{Glaciated: } land covered in the past by any form of glacier is said to be glaciated
\vocab{Glacier: } A river of ice that erodes, transports material, and carves earth's surface. They are formed by years of accumulation of snow that hardens and becomes a glacier.
\vocab{Glacier: } a mass of ice that originates on land, usually having an area larger than one-tenth of a square kilometer; many believe that a glacier must show some type of movement; others believe that a glacier can show evidence of past or present movement
\vocab{Glacier cave: } a cave of ice, usually underneath a glacier and formed by meltwater; cave entrances are often enlarged near a glacier terminus by warm winds; most common on stagnant portions of glaciers
\vocab{Glacieret: } a very small glacier
\vocab{Glacier fire: } a phenomenon in which strong reflection of the sun on an icy surface causes a glacier to look like it is on fire
\vocab{Glacier flood: } a sudden outburst of water released by a glacier
\vocab{Glacier flour: } a fine powder of silt- and clay-sized particles that a glacier creates as its rock-laden ice scrapes over bedrock; usually flushed out in meltwater streams and causes water to look powdery gray; lakes and oceans that fill with glacier flour may develop a banded appearance; also called rock flour
\vocab{Glacier ice: } well-bonded ice crystals compacted from snow with a bulk density greater than 860 kilograms per cubic meter (55 pounds per cubic-foot)
\vocab{Glacierized: } land overlaid at present by a glacier is said to be covered; the alternative term glacierized has not found general favour
\vocab{Glacier Milk: } Flour that has been mixed with water to create milky appearance.
\vocab{Glacier mill: } a nearly vertical channel in ice that is formed by flowing water; usually found after a relatively flat section of glacier in a region of transverse crevasses
\vocab{Glacier pothole: } potholes formed at the bottom of glaciers through erosion caused by sand and gravel in melt-water; melt-water seeps through crevasses in the glaciers, sometimes forming whirlpools; at the bottom of the glacier, the water is under very high pressure, leading to erosion of underlying rocks
\vocab{Glacier remainie: } a glacier that is reconstructed or reconstituted out of other glacier material; usually formed by seracs falling from a hanging glacier, then re-adhering; also called reconstituted, reconstructed or regenerated glacier
\vocab{Glacier snout: } the lowest end of a glacier; also called glacier terminus or toe
\vocab{Glacier sole: } the bottom of the ice of a glacier
\vocab{Glacier table: } a rock that resides on a pedestal of ice; formed by differential ablation between the rock-covered ice and surrounding bare ice
\vocab{Glacier terminus: } the lowest end of a glacier; also called glacier snout or toe
\vocab{Glacier toe: } the lowest end of a glacier; also called glacier snout or terminus
\vocab{Glacier trough: } u-shaped valleys transformed from v-shaped stream valleys due to erosion caused by passing glaciers
\vocab{glaciofluvial: } Geomorphic feature whose origin is related to the processes associated with glacial meltwater.
\vocab{Gleization: } A soil formation process that occurs in poorly drained environments. Results in the development of extensive soil organic layer over a layer of chemically reduced clay that takes on a blue color.
\vocab{Gneiss: } Striped, metamorphic rock formed deep underground
\vocab{Graupel: } Variations in temperature, migration of liquid and vapor water, and pressure of snow cover may result in rounded snow pellets from 2 to 5 mm diameter. Graupel is visually similar to hail, but lacks the banded outward growth pattern of hail.
\vocab{Grooves: } larger striations, created when larger rocks scape bedrock beneath a glacier
\vocab{Gros ventre: } Reverse fault
\vocab{Ground moraine: } continuous layer of till near the edge or underneath a steadily retreating glacier
\vocab{Halocene: } 10,000 years ago-present day
\vocab{Hanging glacier: } a glacier that terminates at or near the top of a cliff
\vocab{Hanging valley: } a valley formed by a small glacier that has a valley bottom relatively higher than nearby valleys formed by larger glaciers
\vocab{Headwall: } a steep cliff, usually the uppermost part of a cirque
\vocab{Hinge fault: } A fault where the foot wall or hanging wall is slanted
\vocab{Hoarfrost: } a deposit of interlocking ice crystals (hoar crystals) formed by direct sublimation on objects, usually those of small diameter freely exposed to the air, such as tree branches, plant stems and leaf edges, wires, poles, etc.; the surfaces of these objects are sufficiently cooled, mostly by nocturnal radiation, to cause the direct sublimation of the water vapor contained in the ambient air.
\vocab{Horn: } a peak or pinnacle thinned and eroded by three or more glacial cirques
\vocab{Hummock: } Small area of raised ground which is formed as a glacier slowly retreats, leaving behind ground moraine.
\vocab{Hypisthermal: } A period of time in the mid-Holocene when climate was generally warmer. Also called the altithermal.
\vocab{Ice apron: } a mass of ice adhering to a mountainside
\vocab{ice breccia: } Large angular ice fragments embedded in finer ice or snow record abrupt changes.
\vocab{Ice cap: } a dome-shaped mass of glacier ice that spreads out in all directions; an ice cap is usually larger than an icefield but less than 50,000 square kilometers (12 million acres)
\vocab{Ice Cap Glacier: } Mounds of ice that submerge peaks and ridges at the crest of a mountain range
\vocab{Ice cave: } a cave of ice, usually underneath a glacier and formed by meltwater; cave entrances are often enlarged near a glacier terminus by warm winds; most common on stagnant portions of glaciers
\vocab{Ice-cemented glacier: } a rock glacier that has interstitial ice a meter or so below the surface
\vocab{Ice-cored glacier: } a rock glacier that has a buried core of ice
\vocab{Ice covered: } land overlaid at present by a glacier is said to be covered; the alternative term glacierized has not found general favor
\vocab{Ice Cubic: } Cubic ice is a solid form of water that has been proposed to form in high clouds. metastable.
\vocab{Ice divide: } the boundary separating opposing flow directions of ice on a glacier or ice sheet
\vocab{Ice Dome: } ice surface with parabolic surface; located in accumulation zone
\vocab{Icefall: } part of a glacier with rapid flow and a chaotic crevassed surface; occurs where the glacier bed steepens or narrows
\vocab{Ice falls: } happen when glaciers flow over a steep drop or squeeze through narrow places
\vocab{Ice field: } a mass of glacier ice; similar to an ice cap, and usually smaller and lacking a dome-like shape; somewhat controlled by terrain
\vocab{Ice Hexagonal: } (ice Ih) is the form of all natural snow and ice on Earth as evidenced in the six-fold symmetry in ice crystals grown from water vapor (that is, snowflakes).
\vocab{Ice quake: } a shaking of ice caused by crevasse formation or jerky motion
\vocab{Ice rise: } when ice gets on top of rock in the seabed, these happen to ice shelves, they are usually dome shaped
\vocab{Ice sheet: } a dome-shaped mass of glacier ice that covers surrounding terrain and is greater than 50,000 square kilometers (12 million acres), the Greenland and Antarctic ice sheets)
\vocab{Ice Shelves: } ice sheet attached to land, extends over sea, floats on water
\vocab{Ice stream: } (1) a current of ice in an ice sheet or ice cap that flows faster than the surrounding ice (2) sometimes refers to the confluent sections of a branched-valley glacier (3) obsolete synonym of valley glaciers
\vocab{Ice Tongue: } a long and narrow sheet of ice projecting out from the coastline to the ocean.
\vocab{ice-wedge casts: } A vertical structure that results from cracks in frozen ground (by means of ice wedging) which are later filled by sediment. They are similar to infilled mudcracks in drying lakes, but usually larger.
\vocab{Igneous dike: } magma forced into a vertical cracks
\vocab{Illinoian: } 300,000-125,000 years ago; North American glaciation related to European Riss glaciation.
\vocab{Indian Ocean, Atlantic Ocean, Pacific Ocean: } the oceans that are all part of the Southern Ocean
\vocab{Interglacial: } A period of time between two glaciations
\vocab{interglacial periods: } Times between recognized advances of the ice. Sea level can be hundreds of feet higher in interglacials than in glacial periods. The present time is the latest interglacial period.
\vocab{Isostatic rebound adjustment: } Up or down warping of the Earth's lithosphere to accommodate for mass being added or removed. Northern Ontario, Canada is rebounding in adjustment to the last glacial retreat 10,000 years ago.
\vocab{James Cook: } the early explorer that was the first to cross the Antarctic Circle
\vocab{Jokulhlaup: } (1) a large outburst flood that usually occurs when a glacially dammed lake drains catastrophically (2) any catastrophic release of water from a glacier
\vocab{Kame: } an irregularly shaped hill or mound composed of sand, gravel and till that accumulates in a depression on a retreating glacier, and is then deposited on the land surface with further melting of the glacier.
\vocab{Kame: } Rounded hill of sand and gravel deposited by meltwater within or at the base of the glacier.
\vocab{Kansan: } North American glaciation related to European Mindel glaciation.
\vocab{katabatic wind: } A wind that flows from a glacier, caused by air cooled by the ice becoming heavier than surrounding air, then draining down-valley.
\vocab{Kettle: } A depression formed by melting glacial ice
\vocab{Kettle Hole: } A circular depression in the ground made when a block of ice calves off the toe of a glacier, becomes buried by till, and later melts.
\vocab{Kettles: } irregular till thickness and depressions where large blocks of ice melted within the till
\vocab{kinematic waves: } These ice waves move downglacier and are propagated by increasing glacial thickness. Kinematic waves may move two to six times the velocity of surrounding, thinner ice.
\vocab{Knife Edged Ridges/ Pointe D Peaks: } ridges between widening u-shaped glacial valleys that become narrower until they rise steeply to narrow, aretes/pointy pyramids
\vocab{krill: } the animal that is the most abundant in the Southern Ocean
\vocab{lacustrine: } Pertaining to lakes. Lacustrine proglacial deposits usually show confined sorting of fine sized sedimentary particles. They may have dropstones if icebergs once floated in the lake. Other lakes associated with glaciers are supraglacial lakes and kettle lakes.
\vocab{Lambert Glacier: } the largest glacier in the world
\vocab{Lateral moraine: } a ridge-shaped moraine deposited at the side of a glacier and composed of material eroded from the valley walls by the moving glacier
\vocab{Lateral moraine: } moraine that forms along the side of a glacier
\vocab{Laurentide Ice Sheet: } The continental glacier that covered eastern Canada and parts of the northeastern United States during the Pleistocene Epoch
\vocab{leads: } Long, narrow openings or fractures in sea ice.
\vocab{Leeward Side: } Side of a natural or man made elements that does not receive wind
\vocab{lens/lenticular: } A thick-in-the-middle/thin-at-the-edges geologic deposit in which the surfaces converge together.
\vocab{Loess: } wind-blown silt deposits blown away from the floodplains and bars of the outwash streams that built up as sand dunes and a frosting of fine silt
\vocab{loesskinder: } Calcarous concretions found in loess deposits. Called "children of the loess" by the Germans who were the first to describe them because the concretions often take pentacular or homunucular proportions.
\vocab{Luis Agassiz: } Proposed that ice ages occurred in the past
\vocab{Marginal crevasse: } a crevasse near the side of a glacier formed as the glacier moves past stationary valley walls; usually oriented about 45 degrees up-glacier from the side wall
\vocab{Mass Balance: } the difference between accumulation levels and ablation
\vocab{Mc Murdo Station: } the largest research station in Antarctica
\vocab{Medial moraine: } A moraine formed when two lateral moraines merged together
\vocab{Meltwater conduit: } a channel within, underneath, on top of, or near the side of a glacier that drains meltwater out of the glacier; usually kept open by the frictional heating of flowing water that melts the ice walls of the conduit
\vocab{Metastable: } theoretically unstable but so long-lived as to be stable for practical purposes.
\vocab{Mindel: } European glaciation related to North American Kansan glaciation.
\vocab{Moraine: } a mound, ridge, or other distinct accumulation of glacial till
\vocab{Moraine shoal: } glacial moraine that has formed a shallow place in water
\vocab{Moulin: } a nearly vertical channel in ice that is formed by flowing water; usually found after a relatively flat section of glacier in a region of transverse crevasses; also called a pothole
\vocab{Mountain glacier: } a glacier that is confined by surrounding mountain terrain; also called an alpine glacier
\vocab{NADW (North Atlantic Deep Water): } The deep portion of the thermohaline circulation in the northern Atlantic Ocean.
\vocab{Nebraskan: } the earliest Pleistocene glaciation for which there is ample evidence for, it occurred about 650,000 years ago before the Kansan glaciations. North American glaciation related to European Gunz glaciation.
\vocab{Negative mill: } a geyser; a fountain that develops when water from a conduit is forced up to the surface of a glacier
\vocab{Neve: } The upper area of accumulation in a glacier where firn is found.
\vocab{Niche glacier: } very small glacier that occupies gullies and hollows on north-facing slopes (northern hemisphere); may develop into cirque glacier if conditions are favorable
\vocab{Normal Stress: } normal stress = density * height * g * cos(angle)
\vocab{Nunatak: } a rocky crag or small mountain projecting from and surrounded by a glacier or ice sheet
\vocab{Ogives: } alternate bands of light and dark ice seen on a glacier surface; Dark= summer, Light=winter. They kind of bend towards the middle. indicates the middle of the glacier flows faster than the sides
\vocab{Outburst flood: } any catastrophic flooding from a glacier; may originate from trapped water in cavities inside a glacier or at the margins of glaciers or from lakes that are dammed by flowing glaciers
\vocab{Outlet glacier: } a valley glacier which drains an inland ice sheet or ice cap and flows through a gap in peripheral mountains
\vocab{outlet glaciers: } Valley glaciers which permit ice to move from accumulation areas through mountainous terrain to the sea.
\vocab{Outwash: } material deposited by meltwater from a glacier
\vocab{Outwash Plain: } Formed when sand is eroded, transported and deposited by meltwater streams from the glaciers snout and nearby till deposits to areas in front of the glacier.
\vocab{paleosol: } An ancient or buried soil, often used as a stratigraphic marker for interglacial periods.
\vocab{pancake ice: } Coherent plates of ice that can reach a few meters across and grow from thickened grease ice and resembles pancakes or lily pads.
\vocab{Paternoster lakes: } a series of tarns connected by a single stream or a braided stream system
\vocab{Patterned grounds: } consists of mostly symmetrical geometries displayed across the ground surface in relation to local frost action and cryogenic processes. Patterns emerge as a result of surface disturbances caused by thermal anomalies and freeze processes such as frost heave. Frost heave will disturb the frost layer as ice lenses accumulate and protrude, causing unstable soil conditions. Can be polygons, circles, stripes, nets, and steps.
\vocab{Periglacial: } relating to or denoting an area adjacent to a glacier or ice sheet or otherwise subject to repeated freezing and thawing
\vocab{periglacial: } The area around a glacier often characterized by harsh climate.
\vocab{Piedmont glacier: } large ice lobe spread out over surrounding terrain, associated with the terminus of a large mountain valley glacier
\vocab{Pingo: } Large mounds of earth-covered ice which form in a permafrost environment which are found in Alaska, Greenland and Antarctica. Pingos may be up to 70 meters tall and 600 meters diameter. The word was borrowed from the Inuit in 1938 by A.E. Porsild after whom Porsild Pingo in Tuktoyaktuk was named. Pingos have an average life-time of about 1,000 years.
\vocab{Pingo: } also called hydrolaccolith or bulgunniakh, is a mound of earth-covered ice found in the Arctic and subarctic that can reach up to 70 metres in height and up to 600 m in diameter.
\vocab{pitch: } In climbing, a unit of measure approximately equal to the length of your rope, or the distance between fixed anchor positions. To go secured from pitch 3 to pitch 4 of a glacier means that you would be protected by a rope and safety gear during that part of the ascent. "Unprotected pitches" include many icefalls and crevasse fields.
\vocab{Plastic Deformation: } When a sufficient load is applied to a material, it will cause the material to change shape. Ice deforms below 60 m, grains within the formation change shape slowly, new grains grow where old ones disappear. This allows the glacier to move.
\vocab{Plastic Flow: } slow movement of a glacier in which ice crystals slip over each other
\vocab{Plastic Zone: } place where cracks cannot form in the glacier
\vocab{Plestocene: } 1.8 million years ago to 11,000 years ago. The Last Ice Age.
\vocab{Plucking the glacier freezing onto masses of rock, and glacier flow causing this mass being pulled and broken off, and carried by the glacier: } 
\vocab{Pluvial Lake: } A lake formed to the south of a continental glacier as a result of enhanced rainfall during an ice age. (Example: Lake Bonneville in Utah)
\vocab{Pluvial Processes: } Glaciers moving sediment because of the water in, on and under the glacier
\vocab{Polar glacier: } a glacier entirely below freezing, except possibly for a thin layer of melt near the surface during summer or near the bed; polar glaciers are found only in polar regions of the globe or at high altitudes
\vocab{Polar plateau: } The relatively flat, elevated central region of the East Antarctic Ice Sheet.
\vocab{polynya: } The open seawater between pack-ice and the land or the edge of a glacier.
\vocab{Pothole: } a nearly vertical channel in ice that is formed by flowing water; usually found after a relatively flat section of glacier in a region of transverse crevasses; also called a moulin
\vocab{Proglacial lake: } Lake that forms in front of a glacier
\vocab{Push moraine: } moraine built out ahead of an advancing glacier
\vocab{Quarternary: } geologic period of the late Cenozoic c. two million years ago to the present. The name refers to the fourth interval of earth time, according to early geologists.
\vocab{Randkluft: } a fissure that separates a moving glacier from its headwall rock; like a bergschrund
\vocab{Raymond Bump: } Beneath ice sheet divides ice stratigraphy often develops arches (Raymond Bump) with amplitudes that record the divide's horizontal residence time.
\vocab{Rcessional Moraine: } Terminal moraine after it has come back and moved forward again.
\vocab{Receding glacier: } When ablation exceeds accumulation and the glacier recedes
\vocab{recessional moraine: } Ground moraine deposited as glacier recedes. It usually forms perpendicular to the glacier
\vocab{Reconstituted glacier: } a glacier that is reconstructed or reconstituted out of other glacier material; usually formed by seracs falling from a hanging glacier then re-adhering; also called reconstructed glacier, regenerated glacier, or glacier remainie
\vocab{Reconstructed glacier: } a glacier that is reconstructed or reconstituted out of other glacier material; usually formed by seracs falling from a hanging glacier then re-adhering; also called reconstituted glacier, regenerated glacier, or glacier remainie
\vocab{Regelation: } motion of an object through ice by melting and freezing that is caused by pressure differences; this process allows a glacier to slide past small obstacles on its bed
\vocab{Regenerated glacier: } a glacier that is reconstructed or reconstituted out of other glacier material; usually formed by seracs falling from a hanging glacier then re-adhering; also called reconstituted or reconstructed glacier, or glacier remainie
\vocab{relative humidity: } Relative humidity is actual humidity of a packet of air divided by maximum possible humidity that air can hold. In glaciers, if the relative humidity of a packet of air is high enough that the air reaches the dew point as it cools in contact with the snow or ice, condensation occurs, releasing +AH4- 680 calories/gram of latent heat.
\vocab{relief: } The vertical difference between the surface in valleys and hilltops or the vertical between the base of a glacier and its top.
\vocab{Retreat: } when a mountain glacier's terminus doesn't extend as far downvalley as it previously did; occurs when ablation surpasses accumulation
\vocab{Retreating glacier: } a glacier whose terminus is increasingly retreating upvalley compared to its previous position due to a higher level of ablation compared to accumulation
\vocab{rheology: } The study of flow behavior and characteristics.
\vocab{Ribbon Lake Long:} thin lakes that form after a glacier retreats that form in hollow 
\vocab{Richard Byrd: } The explorer to first fly over the South Pole
\vocab{Rigid zone: } Upper part of a glacier in which there is no plastic flow.
\vocab{rime: } Ice deposits formed when supercooled water droplets freeze on contact with an object (deposition).
\vocab{Riss: } European glaciation related to North American Illinoian glaciation.
\vocab{Roche Mountain: } Rock drumlin.
\vocab{Rock flour: } a fine powder of silt- and clay-sized particles that a glacier creates as its rock-laden ice scrapes over bedrock; usually flushed out in meltwater streams, causing water to look powdery gray; lakes and oceans that fill with glacier flour may develop a banded appearance
\vocab{Rock glacier: } looks like a mountain glacier and has active flow; usually includes a poorly sorted mess of rocks and fine material; may include: (1) interstitial ice a meter or so below the surface ("ice-cemented"), (2) a buried core of ice ("ice-cored"), and/or (3) rock debris from avalanching snow and rock
\vocab{Rogen Moraine: } (also called ribbed moraine) is a subglacially (i.e. under a glacier or ice sheet) formed type of moraine landform, that mainly occurs in Fennoscandia, Scotland, Ireland and Canada. Landform assemblage of numerous, parallel, closely-spaced ridges consisting of glacial drift, usually TILL. The ridges are formed transverse to ice flow in a subglacial position and are usually found in the central portions of former ice sheets. believed to have been the central areas of the ice sheets. Formation linked closely to Drumlins.
\vocab{Sandar: } Flat outwash plains caused by glacial melting feature braided streams and sinous sand and gravel bars.
\vocab{Sastrugi: } parallel wave-like ridges caused by winds on the surface of hard snow, especially in polar regions.
\vocab{Sedimentary ogives: } alternating bands of light and dark at the firn limit of a glacier; the light bands are usually young and lightest at the highest level up-glacier, becoming increasingly older and darker as they progress down-glacier
\vocab{Serac: } an isolated block of ice that is formed where the glacier surface is fractured
\vocab{shear stress: } shear stress = density * height * g * sin(angle)
\vocab{Sichelwannen- curved grooves formed by water under immense pressure at the base of a glacier: } 
\vocab{Sintering: } the bonding together of ice crystals
\vocab{Snowdrift glacier: } a semipermanent mass of firn formed by drifted snow behind obstructions or in the ground; also called a catchment glacier or a drift glacier
\vocab{Snowline: } The end of the zone of accumulation and start of the zone of melting
\vocab{snow line: } The elevation above which snow remains all year long
\vocab{Solifluction: } A slow, viscous, downslope flow of saturated sediment and rock debris especially in areas underlain by frozen ground.
\vocab{Southern Ocean: } the ocean that surrounds Antarctica
\vocab{South Pole: } the important place located in Antarctica
\vocab{Splay crevasse: } a crevasse pattern that forms where ice slowly spreads out sideways; commonly found near a glacier terminus
\vocab{Stratified Drift: } Sediment laid down by glaial meltwater (lasagna)
\vocab{Striation: } Grooves in bedrock gouged in by moving glaciers
\vocab{Subglacial Volcano: } is a volcanic form produced by subglacial eruptions or eruptions beneath the surface of a glacier or ice sheet which is then melted into a lake by the rising lava. The more distinctly flat-topped, steep-sided subglacial volcanoes are called tuyas, named after Tuya Butte in northern British Columbia by Canadian geologist Bill Mathews in 1947. Often cause Jokulhlaups.
\vocab{Sub polar glacier: } a glacier whose temperature regime is between polar and temperate; usually predominantly below freezing, but could experience extensive summer melt
\vocab{Sun cup: } Ablation hollow that develops, during intense sunshine.
\vocab{Surface Hoar: } Deposition of ice crystals on a surface which occurs when the temperature of the surface is colder than the air above and colder than the frost point of that air.
\vocab{Surges: } in the summer months when the glacier is melting there will be quick sunspots
\vocab{Surging glacier: } a glacier that experiences a dramatic increase in flow rate, 10 to 100 times faster than its normal rate; usually surge events last less than one year and occur periodically, between 15 and 100 years
\vocab{Tabular: } Thinner in one direction than the other
\vocab{Tarn: } a small mountain lake or pool; a mountain lake formed in a cirque excavated by a glacier. A moraine may form a natural dam below a tarn
\vocab{temperature-gradient metamorphism: } Process of firnification when large temperature gradients exist within the snowpack, such as within adjacent layers of snow. When all snow is fully converted to ice, the density is about 830-840 kg/cubic meter.
\vocab{terminal/end moraine: } Moraine deposited at the end of a glacier
\vocab{Terminus: } End of glacier
\vocab{Terminus: } the lowest end of a glacier, also called the glacier toe or glacier snout
\vocab{Teton blockfault: } One side bound by a normal fault on one side
\vocab{Teton Fault: } Normal fault
\vocab{thermohaline circulation: } is a part of the large-scale ocean circulation that is driven by global density gradients created by surface heat and freshwater fluxes.
\vocab{Thomson crystal: } a large ice crystal found in deep, stagnant water-filled cavities of a glacier
\vocab{Tidewater glacier: } mountain glacier that terminates in the ocean
\vocab{Till: } Any kind of glacial deposit
\vocab{Till: } The sediments deposited directly by a glacier;
\vocab{Tongue: } a projection of the ice edge up to several km in length caused by wind and current; usually forms when a valley glacier moves very quickly into a lake or ocean
\vocab{Tor: } A high rock, a high rocky hill, or pile of rocks.
\vocab{Trans antarctic mountains: } the natural border between East and West Antarctica
\vocab{Tributary glacier: } a small glacier that flows into a larger glacier
\vocab{Tuya: } Volcanos which erupted beneath glaciers, melts through the ice above and finishes with a subaerial lava flow are called "tuyas" after the type locality, Tuya Butte, Stikine Belt, Northwest British Columbia, Canada. Also called "table mountains," the typical flat-top is created by the explosive interaction of hot magmas and lavas with water and ice. Tuyas reveal the height of the ice at the time of eruption and may be used as part of correlative dating. Similar volcanic structures, called "guyots" or "sea mounts," are created under the oceans when magma hits ocean water.
\vocab{U-shaped valley: } The shape of a valley formed by the erosion of a glacier
\vocab{Valley glacier: } a mountain glacier whose flow is confined by valley walls
\vocab{Varve: } A pair of thin layers deposited during a single year in a lake. One layer consists of silt brought in during spring floods and the other of clay deposited in winter when the lake's surface freezes over and the water is still.
\vocab{Ventifact: } A faceted or triangular-shaped stone formed by sandblasting in strong winds, often highly polished.
\vocab{Vienna Standard Mean Ocean Water (VSMOW): } is a water standard defining the isotopic composition of fresh water.
\vocab{Vinson Massif: } highest mountain peak in Antarctica
\vocab{Wave ogives: } ogives that show some vertical relief on a glacier; usually the dark bands are in the hollows and the light bands are in the ridges; form at the base of steep, narrow ice falls
\vocab{Weathered ice: } glacier ice that has been exposed to sun or warm wind so that the boundaries between ice crystals are partly disintegrated
\vocab{Weddell Sea: } the sea that is north of Antarctica
\vocab{Wet Bottom Glacier: } A glacier with a thin layer of water at its base, over which the glacier slides, also called a temperate glacier. Occur in regions where atmospheric temperatures become warm enough for glacial ice to be at or near its melting point.
\vocab{whale: } animal at the top of the Antarctic food web
\vocab{Whaleback: } similar to a roche moutonnée, but on a larger scale.
\vocab{Where do you find trees and why?: } You find trees at glacial moraines because it has clay, which retains water
\vocab{White-Out: } A weather condition in which the horizon cannot be identified and there are no shadows. White snow blends everywhere. All you see is white. (I've been there before, it's scary. Michaelene Cutro)
\vocab{Wurm: } European glaciation related to North American Wisconsinan glaciation.
\vocab{Zone of Ablation: } In the lower part of a glacier where there is more ablation than accumulation
\vocab{Zone of Accumulation: } In the upper part of the glacier where there is more accumulation than ablation
\vocab{zone of accumulation: } The part of a glacial system where snow and ice are accumulating faster than they are melting away.
\vocab{Zone of Fracture: } The upper 50 meters of glacial ice is brittle and is carried by the ice below it.
\vocab{Zone of plastic flow: } The lower part of the glacier that has the ability to flow due to pressure
\vocab{Zone of Wastage: } below the snow line, where snow melting exceeds snow accumulation
