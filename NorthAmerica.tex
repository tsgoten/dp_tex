\mysection{History 3 major ice sheets} 
	present in North America at various times: \blue{Cordilleran}, \blue{Laurentide} and \blue{Innutitian} or \blue{Franklin} ice sheets. Presence of ice sheet would have \textbf{pushed circulation patterns south}, resulting in increased precipitation in the southern portion of the US. Minnesota/Nebraska had lobate ice flow over multiple times, leaving rolling hills, drumlins, eskers. Wisconsin had multiple phases of glacial outburst floods from the Columbia Gorge and Glacial Lake Missoula, leaving rough “scabland” topography. 
	\mysub{Lake Agassiz} – large glacial lake extending from Saskatchewan to Minnesota. The Great Lakes are a remnant of this large inland seaway. Why do plains states grow crops well? Glaciers and ice sheets leave behind till and fine nutrient-rich glacial flour. Much of the soil in the Midwest was blown in from Canada. Causes of glaciation at end of Ordovician - CO2 drop (decrease in global temperatures), rapid sea level change (drop = more land for ice), shifts in circulation (bringing cold air further north) shifts in orbital parameters, continental configuration (allowed for growth of ice sheets) 
	\mysub{Background}
		The major glacial stages of the current ice age in North America are the \blue{Illinoian}, \blue{Eemian} and \blue{Wisconsin} glaciation. The use of the Nebraskan, Afton, Kansan, and Yarmouthian stages to subdivide the ice age in North America has been discontinued by Quaternary geologists and geomorphologists. These stages have all been merged into the Pre-Illinoian in the 1980s
	\mysub{Wisconsin}
		This Wisconsin glaciation left widespread impacts on the North American landscape. The Great Lakes and the Finger Lakes were carved by ice deepening old valleys. Most of the lakes in Minnesota and Wisconsin were gouged out by glaciers and later filled with glacial meltwaters. The old Teays River drainage system was radically altered and largely reshaped into the Ohio River drainage system. Other rivers were dammed and diverted to new channels, such as Niagara Falls, which formed a dramatic waterfall and gorge, when the waterflow encountered a limestone escarpment. Another similar waterfall, at the present Clark Reservation State Park near Syracuse, New York, is now dry.
	\mysub{Glacial Till in NA}
		The area from Long Island to Nantucket, Massachusetts was formed from glacial till, and the plethora of lakes on the Canadian Shield in northern Canada can be almost entirely attributed to the action of the ice. As the ice retreated and the rock dust dried, winds carried the material hundreds of miles, forming beds of loess many dozens of feet thick in the Missouri Valley. Post-glacial rebound continues to reshape the Great Lakes and other areas formerly under the weight of the ice sheets.
	\mysub{Late Glacial Climate Warming}
		\ddd  Climate amelioration began to occur rapidly throughout Western Europe and the North European Plain c. 16,000-15,000 years ago. The environmental landscape became increasingly boreal, except in the far north, where conditions remained arctic. Sites of human occupation reappeared in northern France, Belgium, northwest Germany, and southern Britain between 15,500 and 14,000 years ago. Many of these sites are classified as Magdalenian, though other industries containing distinctive curved back and tanged points appeared as well. As the Fennoscandian ice sheet continued to shrink, plants and people began to repopulate the freshly deglaciated areas of southern Scandinavia.
		\ddd Between 12,000 and 10,000 years ago, the western coast of Norway and southern Sweden to latitude 65° north became occupied by sites belonging to the Fosna-Hensbacka complex. They are defined by the appearance of tanged points and other artifacts similar to those found earlier in Northwest Germany. Komsa sites, dated to about 7,000 years ago, are found along Norway's Finnmark county above 70° north and further east on the Kola Peninsula. They are defined by surface finds of tanged points, burins, scrapers, and adzes. The primary game of Magdalenian hunters appears to have been reindeer, though evidence of bird and shellfish consumption persist, as well.
\mysection{Pleistocene Glaciations}
		\ddd Japan glaciation During the Pleistocene Epoch Ice Age, beginning about 2.5 million years ago, virtually all of southwestern Canada was repeatedly glaciated by ice sheets that also covered much of Alaska, northern Washington, Idaho, Montana, and the rest of northern United States. In North America, the most recent glacial event is the Wisconsin glaciation, which began about 80,000 years ago and ended around 10,000 years ago. [Source: U.S. National Park Service Website, Ice Age Floods, 2002]
		\ddd During the \blue{Great Ice Age, or Pleistocene Epoch}, which began about 2 million years ago, large portions of Canada and the Northern United States were blanketed by the continental ice sheet. Much of the rich soil of the Midwest is glacial in origin, and the drainage patterns of the Ohio River and the position of the Great Lakes were influenced by the ice. The effects of the glaciers can be seen in the stony soil of some areas, the hilly land surfaces dotted with lakes, the scratched and grooved bedrock surfaces, and the long, low ridges composed of sand and gravel which formed at the front of the ice sheet. [Source: Schlee, Our Changing Continent: USGS General Interest Publication, Online, January 2001
		\ddd Increased rainfall in the area south of the continental ice sheet formed large lakes in Utah, Nevada, and California. Remnants of these ancient lakes still exist today as the Great Salt Lake, Pyramid Lake, Winnemucca, and many others. Ancient shorelines for these old lakes can be found along the sides of mountains, as for example, near Provo, Utah.
		\ddd The tremendous size of the ice sheet further influenced paleogeography by lowering sea level about 450 feet below the present level; the water contained in the ice and snow came from the oceans. The continental shelves around our continent, as well as the other continents of the world, were above water and, as a result, some States such as Florida were much larger than they are today. The shoreline deposits and shells at the edge of the Continental Shelf, in waters to 450 feet deep, are evidence of this marked drop in sea level during the Pleistocene.
		\ddd During the Pleistocene Epoch Ice Age, beginning about 2.5 million years ago, virtually all of southwestern Canada was repeatedly glaciated by ice sheets that also covered much of Alaska, northern Washington, Idaho, Montana, and the rest of northern United States. In North America, the most recent glacial event is the Wisconsin glaciation, which began about 80,000 years ago and ended around 10,000 years ago.
\mysection{Fraser Glaciation}
	\ddd Antarctic Ice Age glaciation During the Fraser (Late Wisconsin) Glaciation, the Cordilleran ice sheet advanced southward from source areas in British Columbia and terminated in the United States between the Pacific Ocean and the Continental Divide. The ice sheet extended farthest along major south-trending valleys and lowlands that traverse the international boundary; it formed several composite lobes segregated by highlands and mountain ranges. Each lobe dammed sizable lakes that drained generally southward or westward along ice margins and across divides.
	\ddd During the Fraser Glaciation the Cordilleran ice sheet occupied parts of the Fraser and Puget lowland and Strait of Juan de Fuca between about 18,000 and 13,000 B.P., after the maximum stand of nearby alpine glaciers. At its maximum extent about 14,500 to 14,000 years B.P., the ice-sheet surface sloped from about altitude 1,000 meters at the international boundary to between 0 and 300 meters at the ice terminus on the continental shelf and in the southern Puget lowland. Drainage from deglaciated alpine valleys in the Cascade Range and Olympic Mountains flowed southward along both ice margins and coalesced into meltwater streams that built broad outwash trains southward and westward to the Pacific Ocean. In the North Cascades Range, Cordilleran ice overrode high divides and inundated major drainage basins. The ice-sheet surface descended from above 2,600 meters near the international boundary to 270 meters in the Columbia River valley.
	\ddd East of the Cascade Range, the Okanogan lobe extended southward as a broad lobe that dammed the Columbia River valley to form glacial Lake Columbia. The lake discharged along the course of the Grand Coulee, whose tandem gorges developed by recession of great cataracts beneath catastrophic floods from glacial Lake Missoula.
	\ddd The Columbia River lobe dammed the Spokane valley to form a shallow glacial Lake Spokane. The Pend Oreille River sublobe, and eastern appendage of the Columbia River lobe, was less extensive than formerly inferred. The Priest River valley remained unglaciated except for a distributary of the Purcell Trench lobe that dammed the valley mouth.
	\ddd The Purcell Trench lobe dammed the 2,000-cubic-kilometer glacial Lake Missoula, which successively discharged as huge jökulhlaups that flowed to Spokane along the Rathdrum valley and from upper Pend Oreille River valley. From Spokane the great floods swept across the Channeled Scabland and down the Columbia River valley.
	\mysub{The West Kootenai} and East Kootenai glaciers flowed across a high-relief landscape, terminating within a general upland. The Flathead lobe was more extensive than formerly inferred. Both the Flathead lobe and nearby alpine glaciers reached near-maximum positions during high stands of Lake Missoula and thus during the maximum stand of the Purcell Trench lobe. Topographic lows trending south from southern British Columbia fed each of the major lobes of the Cordilleran ice sheet east and west of the Cascade Range, but the secondary lobation of the ice margins was determined by the configuration of local valleys.
	\ddd \mysub{As the Puget lobe} retreated northward, ice-marginal streams and proglacial lakes progressively expanded northward. Glacial Lake Russell drained southward during initial retreat; glacial Lake Bretz later drained northward. Calving into seawater, the Juan de Fuca lobe retreated rapidly and perhaps thereby caused the northwestern part of the Puget lobe to stagnate. Continued ice retreat permitted the sea to enter Puget Sound, and a glaciomarine interval ensued from 13,500 to 11,500 years B.P. Stillstands or readvances of the ice margin occurred during and near the end of the glaciomarine interval. In the northeastern Cascade Range and Waterville Plateau, deglaciation occurred by progressive downwasting and backwasting of ice whose margins frequently stagnated. Most lobes east of the Cascade Range built one or more small recessional moraines. As ice tongues retreated, glacial Lakes Columbia and Missoula fell to successively lower levels as they grew northward behind retreating ice. At length the early lakes were succeeded by glacial Lakes Brewster, Clark, and Kootenay.
	\ddd The apparent absence of the Glacier Peak layer-G tephra within the northern part of its projected fallout area along with the occurrence of several jökulhlaups from glacial Lake Missoula after the Mount St. Helens set-S airfall suggest that much of the glaciated terrain east of the Cascade Range remained glaciated until about 13,000 years ago. In the North Cascade Range, erratics transported by the ice sheet up valleys to cirque floors indicate that, as the ice sheet disappeared, alpine glaciers did not rejuvenate much below the limits of modern glaciers. Although ice lobes both east and west of the Cascade Range generally retreated from terminal positions to the international boundary during the interval 14,000 to 11,000 years B.P., the lobes were not exactly in phase with each other. Particular stillstands and retreats were influenced by local conditions such as topography or seawater that did not affect all lobes equally.
	\mysection{Wisconsin Glaciation}
	\ddd During the Pleistocene Epoch Ice Age, beginning about 2.5 million years ago, virtually all of southwestern Canada was repeatedly glaciated by ice sheets that also covered much of Alaska, northern Washington, Idaho, Montana, and the rest of northern United States. In North America, the most recent glacial event is the Wisconsin glaciation, which began about 80,000 years ago and ended around 10,000 years ago. [Source: U.S. National Park Service Website, Ice Age Floods, 2002
	\ddd A mere 15,000 years ago, during the Ice Age, most of northern America lay under the grip of colossal ice sheets. The effects of the advancing and retreating glaciers can be seen in the headlands of Cape Cod, the Finger Lakes of New York, and the hills of Michigan, but nowhere is the glacier's mark upon the land more impressive than in Wisconsin. Indeed, the state has lent its name to the most recent series of glacial advances and retreats -- the Wisconsin Glaciation lasting from about 100,000 to 10,000 years ago.