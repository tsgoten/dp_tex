\mysection{Ice Cores} a core sample drilled from the accumulation of snow and ice over many years that have recrystallized and have trapped air bubbles from previous time periods, the composition of which can be used to reconstruct past climates and climate change; typically removed from an ice sheet
        \mysub{Oxygen Isotopes}
            \ddd 2 common isotopes $O^16$ and $O^18$
            \ddd Water w/ 16O is lighter, water with 18O is heavier; 16 tends to evaporate easier, causing 18 accumulate in oceans and 16 to end up in water and ice
            \ddd During constant climatic conditions the 16O lost to evaporation returns to the oceans by rain and streams, so that the ratio of 18O to 16O (18O / 16O) is constant.
            \ddd But, during a glaciation, some of the 16O gets tied up in glacial ice and does not return to the oceans. Thus during glaciations the 18O / 16O ratio of sea water increases.
            \ddd During an interglaciation, on the other hand, the 16O that was tied up in glacial ice returns to the oceans causing a decrease in the 18O / 16O ratio of seawater.
            \ddd Thus, we expect that during glaciations the 18O / 16O ratio in seawater will be high, and during interglaciations the 18O /16O ratio in seawater will be low.
        \mysub{Info from Ice Cores}
            \ddd \textbf{Accumulation rate} - The thickess of the annual layers in ice cores can be used to derive a precipitation rate (after correcting for thinning by glacier flow). Past precipitation rates are an important palaeoenvironmental indicator, often correlated to climate change, and it’s an essential parameter for many past climate studies or numerical glacier simulations.
            \ddd \textbf{Melt Layers} - Ice cores provide us with lots of information beyond bubbles of gas in the ice. For example, melt layers are related to summer temperatures. More melt layers indicate warmer summer air temperatures. Melt layers are formed when the surface snow melts, releasing water to percolate down through the snow pack. They form bubble-free ice layers, visible in the ice core
            \ddd \textbf{Past air temperatures} - It is possible to discern past air temperatures from ice cores. This can be related directly to concentrations of carbon dioxide, methane and other greenhouse gasses preserved in the ice.
           	\mysubsub{Rate of flow: } Controlled by: 1. The severity of the slope 2. Basal water= wet bottom= faster flow 3. location within glacier=greater velocity in ice center
        \mysection{Sedimentary Sequences}
        \ddd Sedimentary environments are areas where sediments are deposited; glaciers are an example of this
        \mysub{Supraglacial (ice marginal)}
            \ddd Readily be observed along glacial margins
            \ddd A dark, dirty-ice zone is not uncommon at a glacier’s leading edge
            \ddd The supraglacial environment is a very unstable place because material deposited on top of ice is going to move when the ice melts
            \ddd Till-like mixtures of material with a wide range of particle sizes, called \vocab{"diamicton"}
            \ddd Reflect a complex history of deposition
        \mysub{Subglacial}
            \ddd Most difficult to observe. Rely on ice cores and down-hole cameras 
            \ddd glaciers grind up and mix rock and soil debris in and beneath their base forming a mixture of material (rocks, sand, silt, and clay) that is called till
            \ddd \vocab{Till} is the most common subglacial deposit, but river and lake deposits also occur 
        \mysub{Proglacial}
            \ddd even more dynamic than the subaglacial one
            \ddd glacial meltwater and summer rains carry debris away from the glacier or deposit it in lakes that come and go as the force of the water causes natural dams to give way and lakes to drain, sometimes catastrophically sweeping material away in the wate
            \ddd Include materials sorted by water or wind, river sediment (called outwash), lake sediment, windblown sand, and windblown silt called \vocab{loess} \ddd