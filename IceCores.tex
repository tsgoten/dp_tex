\mysection{Ice Cores} a core sample drilled from the accumulation of snow and ice over many years that have recrystallized and have trapped air bubbles from previous time periods, the composition of which can be used to reconstruct past climates and climate change; typically removed from an ice sheet
        \mysub{Warming and Cooling Periods}
        	Oxygen primarily occurs in two isotopes, 18O and 16O ; water containing 18O is heavier, and thus condenses more readily ; water evaporates from the oceans near the equator and begins to move towards the poles, and condenses and precipitates along the way ; during cooler periods, condensation occurs even closer to the equator  and the vapor that reaches the poles is more depleted in 18O ; during warmer periods, condensation occurs closer to the poles  and the vapor that reaches the poles is richer in 18O.
        \mysub{Oxygen Isotopes}
            \ddd 2 common isotopes $O^16$ and $O^18$
            \ddd Water w/ 16O is lighter, water with 18O is heavier; 16 tends to evaporate easier, causing 18 accumulate in oceans and 16 to end up in water and ice
            \ddd During constant climatic conditions the 16O lost to evaporation returns to the oceans by rain and streams, so that the ratio of 18O to 16O (18O / 16O) is constant.
            \ddd But, during a glaciation, some of the 16O gets tied up in glacial ice and does not return to the oceans. Thus during glaciations the 18O / 16O ratio of sea water increases.
            \ddd During an interglaciation, on the other hand, the 16O that was tied up in glacial ice returns to the oceans causing a decrease in the 18O / 16O ratio of seawater.
            \ddd Thus, we expect that during glaciations the 18O / 16O ratio in seawater will be high, and during interglaciations the 18O /16O ratio in seawater will be low.
        \mysub{Densityof Ice for Air to be trapped: }
        	Firn is not dense enough to prevent air from escaping; but at a density of about \blue{830 kg/m3} it turns to ice, and the air within is sealed into bubbles that capture the composition of the atmosphere at the time the ice formed.
        \mysub{Info from Ice Cores}
            \ddd \textbf{Accumulation rate} - The thickess of the annual layers in ice cores can be used to derive a precipitation rate (after correcting for thinning by glacier flow). Past precipitation rates are an important palaeoenvironmental indicator, often correlated to climate change, and it’s an essential parameter for many past climate studies or numerical glacier simulations.
            \ddd \textbf{Melt Layers} - Ice cores provide us with lots of information beyond bubbles of gas in the ice. For example, melt layers are related to summer temperatures. More melt layers indicate warmer summer air temperatures. Melt layers are formed when the surface snow melts, releasing water to percolate down through the snow pack. They form bubble-free ice layers, visible in the ice core
            \ddd \textbf{Past air temperatures} - It is possible to discern past air temperatures from ice cores. This can be related directly to concentrations of carbon dioxide, methane and other greenhouse gasses preserved in the ice.
           	\mysubsub{Rate of flow: } Controlled by: 1. The severity of the slope 2. Basal water= wet bottom= faster flow 3. location within glacier=greater velocity in ice center
        \mysub{Brittle Ice}
        	Over a depth range known as the brittle ice zone, bubbles of air are trapped in the ice under great pressure. When the core is brought to the surface, the bubbles can exert a stress that exceeds the tensile strength of the ice, resulting in cracks and spall. At greater depths, the air disappears into clathrates and the ice becomes stable again. At the WAIS Divide site, the brittle ice zone was from 520 m to 1340 m depth.
        \mysub{Isotopic Analysis: }
        	The isotopic composition of the oxygen in a core can be used to model the temperature history of the ice sheet. Oxygen has three stable isotopes, 16O, 17O and 18O.[65] The ratio between 18O and 16O indicates the temperature when the snow fell.[66] Because 16O is lighter than 18O, water containing 16O is slightly more likely to turn into vapour, and water containing 18O is slightly more likely to condense from vapour into rain or snow crystals. At lower temperatures, the difference is more pronounced. The standard method of recording the 18O/16O ratio is to subtract the ratio in a standard known as standard mean ocean water (SMOW): \\
        	$ \delta^{18}O = \left(\frac{\left(\frac{18O}{16O}\right) sample}{\left(\frac{18O}{16O}\right) SMOW} - 1\right) \times 1000 per thousand$
        	\mysubsub{Ratios are defined as: }
        		\ddd 2H/1H = 155.76 ppm \ddd 3H/1H = $ 1.85\times10^{-11}  $ \ddd 18O/16O = 2005.2 ppm \ddd 17O/16O = 379.9 ppm
        \mysub{Radiocarbon Dating: }
        	Radiocarbon dating can be used on the carbon in trapped CO2. In the polar ice sheets there is about 15–20 µg of carbon in the form of CO2 in each kilogram of ice, and there may also be carbonate particles from wind-blown dust (\vocab{loess}). The CO2 can be isolated by subliming the ice in a vacuum, keeping the temperature low enough to avoid the loess giving up any carbon. The results have to be corrected for the presence of 14C produced directly in the ice by cosmic rays, and the amount of correction depends strongly on the location of the ice core. Corrections for 14C produced by nuclear testing have much less impact on the results.[43] Carbon in particulates can also be dated by separating and testing the water-insoluble organic components of dust. The very small quantities typically found require at least 300 g of ice to be used, limiting the ability of the technique to precisely assign an age to core depths
    \mysection{Methods of studying glaciers}
	    \ddd The two main processes used to determine ablation or accumulation are \textbf{probing} and \textbf{crevasse stratigraphy}, which can give accurate measurements of snowpack thickness.
	    \mysub{Probing}: researchers will place poles in the icepack at various points, at the beginning of the melt period or accumulation period. After a few months the researchers will return and look at the changes in levels of ice, by looking at the height of the ice along the pole.
	    \mysub{Crevasse stratigraphy}: researchers will find crevasses, then observe the number of layers that formed. Based on the layers the researchers will be able to determine how much snow accumulated. The layers are almost like layers in a tree trunk.
	    \mysub{Cosmogenic nuclide dating} is useful for directly dating rocks on the Earth’s surface. It gives an Exposure Age: that is, how long the rock has been exposed to cosmic radiation. It is effective on timescales of several millions of years. It assumes that boulders have not been buried and then re-exposed at the Earth’s surface.
	    \mysub{Radiocarbon dating} dates the decay of Carbon-14 within organic matter. Organic matter needs to have been buried and preserved for this technique. It is effective for up to the last 40,000 years. It assumes that organic material is not contaminated with older radiocarbon (which, for example, is a common problem with organic material from marine sediment cores around Antarctica).
	    \mysub{Amino Acid Racemisation} dates the decay and change in proteins in organisms such as shells.
	    \mysub{Optically Stimulated Luminescence} dates the radiation accumulated in quartz or feldspar grains within sand. The radiation emanates from radioactive grains within the sediment, such as zircons. It is effective for hundreds of thousands of years, and dates how long the sediment has been buried.
	\mysection{Instruments}
		\mysub{Depth vs. Resolution}
			using higher frequency radio waves gives better resolution, but the radar is more strongly absorbed by ice so the pulses cannot travel as deeply into the ice; conversely using lower frequency waves leads to lower resolution but the waves are not absorbed as strongly and thus penetrate more deeply. 
		\mysub{Raymond Bumps}
			Ice rises situated in the ice-shelf belt around Antarctica have a spatially confined  flow regime, which is decoupled from the surrounding ice shelves, and which contains local ice divides. Beneath the divides, ice stratigraphy often develops arches (Raymond Bumps) with amplitudes that record the divide’s horizontal residence time as well as surface elevation changes. 
    \mysection{Sedimentary Sequences}
        \ddd Sedimentary environments are areas where sediments are deposited; glaciers are an example of this
        \mysub{Supraglacial (ice marginal)}
            \ddd Readily be observed along glacial margins
            \ddd A dark, dirty-ice zone is not uncommon at a glacier’s leading edge
            \ddd The supraglacial environment is a very unstable place because material deposited on top of ice is going to move when the ice melts
            \ddd Till-like mixtures of material with a wide range of particle sizes, called \vocab{"diamicton"}
            \ddd Reflect a complex history of deposition
        \mysub{Subglacial}
            \ddd Most difficult to observe. Rely on ice cores and down-hole cameras 
            \ddd glaciers grind up and mix rock and soil debris in and beneath their base forming a mixture of material (rocks, sand, silt, and clay) that is called till
            \ddd \vocab{Till} is the most common subglacial deposit, but river and lake deposits also occur 
        \mysub{Proglacial}
            \ddd even more dynamic than the subaglacial one
            \ddd glacial meltwater and summer rains carry debris away from the glacier or deposit it in lakes that come and go as the force of the water causes natural dams to give way and lakes to drain, sometimes catastrophically sweeping material away in the wate
            \ddd Include materials sorted by water or wind, river sediment (called outwash), lake sediment, windblown sand, and windblown silt called \vocab{loess} \ddd