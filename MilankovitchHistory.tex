\mysection{Milankovitch Cycles}
         describe the collective effects of changes in the Earth's movements on its climate over thousands of years. variations in \textit{eccentricity}, \textit{axial tilt}, and \textit{precession} of the Earth's orbit resulted in cyclical variation in the solar radiation reaching the Earth, and that this \textit{orbital forcing} strongly influenced climatic patterns on Earth.
         \mysub{Eccentricity:} refers to the earths orbit and its shift from being circular to more elliptical over time;
        \mysub{Axial Tilt (Obliquity): }Tilt of earths axis of rotation. A greater tilt means more drastic seasons; The angle varies between $22.1^\circ$ and $24.5^\circ$, over a cycle of about 41,000 years. The current tilt is $23.44^\circ$. We are currently on a downward trend, meaning warmer climates.
        \mysub{Axial Precession: } is a gravity-induced, slow, and continuous change in the orientation of an astronomical body's rotational axis. The cycle is relative to fixed starts, with a period of $25771.5$ yrs
        \mysub{Aspidal Precession: }changing of the line between the sun and the earth that changes. Tilt of the orbit itself. 
        \mysubsub{Solar Forcing: }changes in these movements of the Earth, which alter the amount and location of solar radiation reaching the Earth.
        \vocab{Perihelion: }closest to the sun; \vocab{Aphelion: }farthest from the sun; The semi-major axis is a constant, therefore when the earth orbit becomes more eccentric, the semi-minor axis shortens. Increase in \vocab{solar irradiation: }at closest approach to the Sun (perihelion) compared to the irradiation at the furthest distance (aphelion) is slightly larger than four times the eccentricity.
        \vocab{Milutin Milankovic} Serbian geophysicsicts and astronomer.
        \ddd \mysub{how long: }100,000 year long cycle. 
    %History of Ice on Earth
    \mysection{History of Ice on Earth}
    	\mysub{Recent History}
    	There have been five or six major ice ages in the past 3 billion years
    	The Late Cenozoic Ice Age began 34 million years ago, its latest phase being the Quaternary glaciation, in progress since 2.58 million years ago. 
        \mysub{Neoproterozoic Snowball on Earth} 
            \vocab{Snowball earth} around \textbf{650 mya}--biological activity in the ocean surface collapsed for millions of years; \textbf{Ended} when volcanic outgassing raised CO2 to 350x modern level; Ocean was virtually covered by thin sea ice + continents were covered in patchy ice due to hydrologic cycle; \vocab{Sir Douglas Mawson} proposed this. 
        \mysub{Late Palezoic Ice Ages}
            \ddd Conventional view: paleozoic ice age was a long ice age for 10 million years w/ some internal waning + waving of glaciers
            \ddd Recent research: series of shorter glacial events separated by periods of warmth
            \ddd Expanded from South America to southern Africa to Australia 
            \ddd The ending constitutes turnover to greenhouse state
            \ddd Sea level response (glacio eustatic) to ice age may be less extreme than once thought
        \mysub{Eocene Oligocene Transition} and the impact of opening oceanic seaways
            \ddd Marked by large scale extinction
            \ddd Most affected organisms were marine or aquatic in nature
            \ddd Major cooling on land and in ocean
            \ddd Causes include volcanic activity + meteorite impacts + decrease in atmospheric CO2
            \ddd Sea level changes mark transition-- in NE Italy, sea level fell 20 m and then 50-60 m in the Oligocene Isotope Event
            \ddd Extinctions could have been caused by volcanic explosions or meteorites
            \ddd Extinction caused by climate change and major fall in sea levels
        \mysub{Pleistocene onset of Northern Hemisphere glaciation}
            \ddd lead to reorganization and relocation of species associations and may have enhanced species turnover
            \ddd Changes in CO2 could have helped to lead to glaciation
            \ddd Began a unique period in Earth’s history where both poles have remained ice locked
            \ddd Between 10 and 6 Ma but did not gain momentum until 3.5-3 Ma
            \ddd Northern Hemisphere glaciation occurred in episodes after Greenland froze
            \ddd Tectonic changes might have triggered more extensive NH glaciation
        \fig{ice_ages1}
        \mysub{Glacial periods} are characterized with large ice sheets and are normally known as ice ages. The periods between these are known as interglacial periods and currently we are in the Holocene interglacial period. The last glacial period was between 120,000 to 11,500 years ago and was during the Pleistocene Epoch.
        \ddd Glacial periods are times in the Earth's history where average global temperatures were approximately 6 C lower and glaciers covered much of the planets surface. The last of these periods ended approximately 10,000 years ago. There are 6 main factors that contribute to global climate and can cause glacial periods: \vocab{solar variability, insulation, dust, atmospheric composition, ocean current circulation, sea ice, and atmospheric circulation.} All of these are natural processes and the only one that is affected by humans is atmospheric composition.
        \ddd \vocab{Interglacial periods} are caused by shifts in the Earth's orbit and this causes a change in the amount of solar radiation that hits the Earth. When the amount of solar radiation increases, this is when the Earth shifts from a glacial to interglacial period. The Quaternary has had multiple shifts between glacial and inter-glacial periods and in the middle of this period, the change cycle between glacial and interglacial changed every 100,000 years. 
        \ddd Antarctica is a great indicator if Earth is in a glacial or interglacial period because the amount of ice and snow on it indicates the amount of solar radiation that is hitting the Earth as well as the average temperature of the Earth.
        \ddd \vocab{CO2 is also an indicator} of the changing from a glacial to interglacial period or vice versa. As the CO2 levels increase the Earth's average temperature will increase and it will move into an interglacial period whereas if the CO2 levels were to fall, the average temperature would fall and the Earth would change to a glacial period.